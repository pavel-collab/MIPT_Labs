\documentclass[a4paper, 12pt]{article}

%%% Работа с русским языком
\usepackage{cmap}					% поиск в PDF
\usepackage{mathtext} 				% русские буквы в формулах
\usepackage[T2A]{fontenc}			% кодировка
\usepackage[utf8]{inputenc}			% кодировка исходного текста
\usepackage[russian]{babel}	% локализация и переносы

%%% Дополнительная работа с математикой
\usepackage{amsmath,amsfonts,amssymb,amsthm,mathtools} % AMS
\usepackage{icomma} % "Умная" запятая: $0,2$ --- число, $0, 2$ --- перечисление

%% Номера формул
%\mathtoolsset{showonlyrefs=true} % Показывать номера только у тех формул, на которые есть \eqref{} в тексте.

%% Шрифты
\usepackage{euscript}	 % Шрифт Евклид
\usepackage{mathrsfs} % Красивый матшрифт

%% Поля
\usepackage[left=2cm,right=2cm,top=2cm,bottom=2cm,bindingoffset=0cm]{geometry}

%% Русские списки
\usepackage{enumitem}
\makeatletter
\AddEnumerateCounter{\asbuk}{\russian@alph}{щ}
\makeatother

%%% Работа с картинками
\usepackage{graphicx}  % Для вставки рисунков
\graphicspath{{images/}{images2/}}  % папки с картинками
\setlength\fboxsep{3pt} % Отступ рамки \fbox{} от рисунка
\setlength\fboxrule{1pt} % Толщина линий рамки \fbox{}
\usepackage{wrapfig} % Обтекание рисунков и таблиц текстом

%%% Работа с таблицами
\usepackage{array,tabularx,tabulary,booktabs} % Дополнительная работа с таблицами
\usepackage{longtable}  % Длинные таблицы
\usepackage{multirow} % Слияние строк в таблице

%% Красная строка
\setlength{\parindent}{2em}

%% Интервалы
\linespread{1}
\usepackage{multirow}

%% TikZ
\usepackage{tikz}
\usetikzlibrary{graphs,graphs.standard}

%% Верхний колонтитул
\usepackage{fancyhdr}
\pagestyle{fancy}

%% Перенос знаков в формулах (по Львовскому)
\newcommand*{\hm}[1]{#1\nobreak\discretionary{}
	{\hbox{$\mathsurround=0pt #1$}}{}}

%% Мои дополнения
\usepackage{float} %Добавляет возможность работы с командой [H] которая улучшает расположение на странице
\usepackage{gensymb} %Красивые градусы
\usepackage{graphicx}               % Импорт изображений
\usepackage{caption} % Пакет для подписей к рисункам, в частности, для работы caption*

\begin{document}
    \begin{center}
    
    \normalsize{Федеральное государственное автономное образовательное учреждение высшего образования}
    
    \textbf{НАЦИОНАЛЬНЫЙ ИССЛЕДОВАТЕЛЬСКИЙ УНИВЕРСИТЕТ \\ <<МОСКОВСКИЙ ФИЗИКО-ТЕХНИЧЕСКИЙ ИНСТИТУТ>>}
    \vspace{13ex}
    
    \textbf{Лабораторная работа 3.1.3\\Измерение магнитного поля Земли.}
    \vspace{40ex}
    
    \normalsize{Филиппенко Павел Сергеевич \\ студент группы Б01-001\\ 2 курс ФРКТ\\}
\end{center}
    
\vfill 
    
\begin{center}
г. Долгопрудный\\ 
2021 г.
\end{center}


\thispagestyle{empty} % выключаем отображение номера для этой страницы
\newpage

    \section*{Цель работы}

    С помощью сцинтилляционного спектрометра исследуется энергетический спектр $\gamma$-квантов, 
    рассеянных на графите. 
    Определяется энергия рассеянных $\gamma$-квантов в зависимости от угла рассеяния, 
    а также энергия покоя частиц, на которых происходит комптоновское рассеяние.

    \section*{Теоретическая чать}

    \section*{Эксперементальная установка}

    Источником излучения служит $^{137}$Cs, испускающий $\gamma$-лучи с энергией 662 кэВ. 
    Он помещен в свинцовый контейнер с коллиматором. 
    Сформированный коллиматором узкий пучок $\gamma$-квантов попадает на графитовую мишень.

    \begin{figure}
        \centering
        \includegraphics[width=\textwidth]{ust.png}
        \caption{Схема эксперементальной установки}
        \label{fig:ust}
    \end{figure}

    Кванты, испытавшие комптоновское рассеяние в мишени, 
    регистрируются сцинтилляционным счетчиком. 
    Счетчик состоит из фотоэлектронного умножителя 3 (далее ФЭУ) и сцинтиллятора 4. 
    Сцинтиллятором служит кристалл NaI(Tl) цилиндрической формы диаметром 40 мм и высотой 40 мм, 
    его выходное окно находится в оптическом контакте с фотокатодом ФЭУ. 
    Сигналы, возникающие на ФЭУ, подаются на ЭВМ для амплитудного анализа. 
    Кристалл и ФЭУ расположены в светонепроницаемом блоке, укрепленном на горизонтальной штанге. 
    Штанга вместе с этим блоком может вращаться относительно мишени, 
    угол поворота отсчитывается по лимбу 6.

    \section*{Обработка эксперементальных данных}

    В формуле для эффекта Комптона 

    \begin{equation}
        \Delta \lambda = \frac{h}{mc} (1 - \cos \theta)
    \end{equation}

    перейдем от длин волн к энергиям фотонов

    \begin{equation}
        \frac{1}{\varepsilon(\theta)} - \frac{1}{\varepsilon_0} = 1 - \cos \theta
    \end{equation}

    где $\displaystyle \varepsilon(\theta) = \frac{E(\theta)}{mc}$ -- приведенная энергия фотона. 
    При этом $m$ -- масса электрона, соответственно $\varepsilon(0) = \varepsilon_0$ -- 
    энергия фотонов, падающих на рассеиватель.

    Теперь заменим в последней формуле приведенную энергию фотона на номер канала $N$, соответсвующего
    вершине фотопика при указанном угле $\theta$.

    \begin{equation}
        \frac{1}{N(\theta)} - \frac{1}{N(0)} = 1 - \cos \theta
    \end{equation}

    \begin{table}[]
    \centering
    \begin{tabular}{|c|c|c|c|c|}
    \hline
    $T_{br}$, $^oC$   & $I$, мкА    & $U$, мВ     & $T$, $^oC$         & $W$, нВт     \\ \hline
    950               & 0,15        & 1,984       & 962,500            & 0,298        \\ \hline
    1000              & 0,15        & 2,150       & 1 016,667          & 0,323        \\ \hline
    1100              & 0,15        & 2,640       & 1 125,000          & 0,396        \\ \hline
    1200              & 0,15        & 3,031       & 1 233,333          & 0,455        \\ \hline
    1300              & 0,15        & 4,008       & 1 341,667          & 0,601        \\ \hline
    1400              & 0,15        & 5,039       & 1 450,000          & 0,756        \\ \hline
    1500              & 0,15        & 6,351       & 1 558,333          & 0,953        \\ \hline
    1600              & 0,15        & 6,948       & 1 666,667          & 1,042        \\ \hline
    1700              & 0,15        & 8,135       & 1 775,000          & 1,220        \\ \hline
    1800              & 0,15        & 8,665       & 1 883,333          & 1,300        \\ \hline
    1900              & 0,15        & 8,925       & 1 991,667          & 1,339        \\ \hline
    \end{tabular}
    \caption{Эксперементальные данные}
    \label{table:1}
\end{table}

    Погрешность угла $\theta$ считаем равным $1^o$, погрешность определения канала $1\%$. В таком случае,
    если $\sigma_{\theta}$ и $\sigma_N$ -- абсолютные погрешности измерения угла и канала соответсвенно,
    то справедливо
    
    \[ \sigma_{1-\cos \theta} = \sin\theta \cdot \sigma_{\theta} \]
    \[ \sigma_{1/N} = \frac{\sigma_N}{N^2} \]

    По эксперементальным данным построим график зависимости $\displaystyle \frac{1}{N}(1 - \cos \theta)$.
    Как видно из формулы, график должен получиться линейным.

    \begin{figure}
        \centering
        \includegraphics[width=\textwidth]{graph.pdf}
        \caption{График зависимости $\frac{1}{N}(1 - \cos \theta)$}
        \label{fig:graph}
    \end{figure}

    Как видно, эксперементальный график и правда линейный. Пересечение этого графика с осью ординат
    есть наилучшее приближение канала при $\theta = 0$, пересечение графика с прямой $1 - \cos \theta = 1$
    есть наилучшее приближение канала при $\theta = 90^o$.

    \begin{center}
        $N_{best}(0) = 898 ~~~ N_{best}(90) =368$
    \end{center}

    Возвращаясь обратно к формуле, содержащей энергии фотонов получем при $\theta = 90^o$

    \begin{equation}
        mc^2 \left (\frac{1}{E(90^o)} - \frac{1}{E(0)} \right) = 1
    \end{equation}

    С учетом, что $E(0) = E_{\gamma} = 662$ кэВ можем получить энергию покоя электрона

    \begin{equation}
        mc^2 = E_{\gamma} \frac{N(90^o)}{N(0) - N(90^o)}
    \end{equation}

    Полученное значение энергии покоя электрона $mc^2 = 459.65$ кэВ. Как мы знаем, табличное значение
    энергии покоя электрона $mc^2 = 511$ кэВ, так что, полученное нами значение несколько отличается от табличного.

\end{document}