\input{preambule_article.tex}

\begin{document}
	\begin{center}
    
    \normalsize{Федеральное государственное автономное образовательное учреждение высшего образования}
    
    \textbf{НАЦИОНАЛЬНЫЙ ИССЛЕДОВАТЕЛЬСКИЙ УНИВЕРСИТЕТ \\ <<МОСКОВСКИЙ ФИЗИКО-ТЕХНИЧЕСКИЙ ИНСТИТУТ>>}
    \vspace{13ex}
    
    \textbf{Лабораторная работа 3.1.3\\Измерение магнитного поля Земли.}
    \vspace{40ex}
    
    \normalsize{Филиппенко Павел Сергеевич \\ студент группы Б01-001\\ 2 курс ФРКТ\\}
\end{center}
    
\vfill 
    
\begin{center}
г. Долгопрудный\\ 
2021 г.
\end{center}


\thispagestyle{empty} % выключаем отображение номера для этой страницы
\newpage
	
	\section*{Задание 1.}
	
	\noindent Соберем на макетной плате схему интегрирующей цепи с параметрами $R = 100$ Ом, $C = 1,05$ мкФ, тогда постоянная времени для этой цепи $\tau = R C = 105$ мкс.
	
	\noindent Экспериментально определим верхнюю граничную частоту $\nu_0$, подбирая $\nu_0$ таким образом, чтобы амплитуда выходного сигнала составила $70 \%$ от амплитуды входного сигнала. $U_{\text{вых}} = 0,7 U_{\text{вх}}$.
	
	\noindent Теоретическое значение $\nu_{\text{0теор}} = \frac{1}{2 \pi C R} = 1,516$ кГц. Экспериментальное значение $\nu_{\text{0экс}} = 1,51$ кГц.
	
	%---------------------------------------------------------------------------------------------------------------------------------------------------------------
	
	\noindent Снимем зависимость коэффициента передачи $K(\nu) = \frac{U_{\text{вых}}}{U_{\text{вх}}}$ от частоты $\nu = 2^n \nu_0$ в пределах $n \in [-2, 4]$. Построим график $K(\nu)$, а так же граф Боде -- $20 \lg K$ от $n = \log_2 (\nu / \nu_0)$.

	\begin{table}[h!]
		\begin{center}
			\begin{tabular}{|c|c|c|c|}
				\hline
				$n$& $\nu$, кГц & $K(\nu)$ &  $20 \lg K$      \\ \hline
				-2 & 0,377      & 0,96     & -0,35  \\ \hline
				-1 & 0,755      & 0,9      & -0,92  \\ \hline
				0  & 1,51       & 0,73     & -2,73  \\ \hline
				1  & 3,02       & 0,55     & -5,19  \\ \hline
				2  & 6          & 0,32     & -9,90  \\ \hline
				3  & 12         & 0,16     & -15,92 \\ \hline
				4  & 24,1       & 0,088    & -21,11 \\ \hline
			\end{tabular}
		\end{center}
		\caption{}
	\end{table}

	% график 1
	% график 2
	% картинка колебаний
	
	%---------------------------------------------------------------------------------------------------------------------------------------------------------------
	
	\noindent По осцилограмме прямоугольных сигналов оценим постоянную времени $\tau$,\\ измерив время нарастания фронта импульса от нуля до уровня $1 - 1/e \approx 0,63$.
	
	\noindent Экспериментально получено $\tau = 111$ мкс. Тогда $\nu_0 = \frac{1}{2 \pi \tau} \approx 1434$ Гц.
	
	% картинка прямоугольных сигналов. 
	
	%---------------------------------------------------------------------------------------------------------------------------------------------------------------
	
	\noindent Соберем на макетной плате схема интегрирующей цепи. Экспериментально определим нижнюю граничную частоту $\nu_0$.
	
	\noindent Теоретическое значение $\nu_{\text{0теор}} = \frac{1}{2 \pi C R} = 1,516$ кГц. Экспериментальное значение $\nu_{\text{0экс}} = 1,9$ кГц.
	
	%---------------------------------------------------------------------------------------------------------------------------------------------------------------
	
	\noindent Снимем зависимость коэффициента передачи $K(\nu) = \frac{U_{\text{вых}}}{U_{\text{вх}}}$ от частоты $\nu = 2^n \nu_0$ в пределах $n \in [-4, 2]$. Построим график $K(\nu)$, а так же граф Боде -- $20 \lg K$ от$n = \log_2 (\nu / \nu_0)$.
	
	
	\begin{table}[h!]
		\begin{center}
			\begin{tabular}{|c|c|c|c|}
				\hline
				n  & $\nu$, Гц & $K(\nu)$ & $20 \lg K$  \\ \hline
				-4 & 118,7     & 0,07     & -23,10 \\ \hline
				-3 & 237,5     & 0,11     & -19,17 \\ \hline
				-2 & 475       & 0,246    & -12,18 \\ \hline
				-1 & 950       & 0,43     & -7,33  \\ \hline
				0  & 1900      & 0,69     & -3,22  \\ \hline
				1  & 3800      & 0,87     & -1,21  \\ \hline
				2  & 7600      & 0,95     & -0,45  \\ \hline
			\end{tabular}
		\end{center}
		\caption{}
	\end{table}
	
	% график 1
	% график 2
	% картинка колебаний

	%---------------------------------------------------------------------------------------------------------------------------------------------------------------

	\noindent Изучим графики частотной и фазовой характеристик интегрирующей цепи в MicroCap. 
	Верхняя частота $f_0 = 9,99 \approx 10$ кГц. Изучим переходную характеристику.

	\noindent Расчитаем постоянную времени $\tau = (R || R_L) \cdot C = 16$ мкс, значение, найденное по графику $\tau = 16,5$ мкс.

	\noindent Убедимся в том, что при наличии сопротивления $R_L$ передаточная функция
	цепи принимает вид

	\begin{equation*}
		H(p) = \frac{K_0}{1 + p \tau}, ~ K_0 = \frac{R_L}{R + R_L}, ~ \tau = (R || R_L) \cdot C
	\end{equation*}

	\begin{figure}[h!]
		\centering
		\includegraphics[scale = 0.7]{MC_task1.1.png}
		\caption{}
	\end{figure}

	%-------------------------------------------------------------------------------------------------------------------------

	\noindent Изучим частотную и фазовую характеристики дифференцирующей цепи. Нижняя частота $f_0 \approx 10$ кГц. Изучим переходную
	характеристику. По графику оценим постоянную времени $\tau \approx 16,9$ мкс. 
	Убедимся, что при $R_s \neq 0$ передаточная функция принимает вид

	\begin{equation*}
		H(p) = \frac{K_0 p \tau}{1 + p \tau}, ~ K_0 = \frac{R}{R + R_s}, \tau = (R + R_s) \cdot C
	\end{equation*}

	\begin{figure}[h!]
		\centering
		\includegraphics[scale = 0.7]{MC_task1.2.png}
		\caption{}
	\end{figure}	

	% =======================================================================================================================
	
	\section*{Задание 2.}

	\begin{equation*}
		f_0 = \frac{1}{2 \pi RC} = 10 ~ \text{кГц}
	\end{equation*}

	\noindent По графикам ФЧХ измерим значения фазовых
	сдвигов ФВЧ, ПФ и ФНЧ на частотах $0, ~ f_0, ~ \infty$.

	\begin{table}[h!]
		\begin{center}
			\begin{tabular}{|c|c|c|c|}
				\hline
							& ФВЧ & ПФ  & ФНЧ0 \\ \hline
				0       	& 180 & 90  & 0    \\ \hline
				$f_0$   	& 90  & 0   & -90  \\ \hline
				$\infty$	& 0   & -90 & -180 \\ \hline
			\end{tabular}
		\end{center}
	\end{table}

	\noindent Заметим, что двухсторонняя полоса пропускания ПФ $\Delta f = 36 - 3 ~ \text{кГц} = 33 ~ \text{кГц} \approx 3f_0$ .

	%-------------------------------------------------------------------------------------------------------------------------

	Открыв графики переходных характеристик, оценим время спада $\tau^{(-)}$ первого выброса
	переходной характеристики ФВЧ до уровня $e^{-1}$ и время $\tau^{(+)}$ нарастания фронта переходной
	характеристики ФНЧ до уровня $1 - e^{-1}$.

	\begin{center}
		\fbox{$\tau^{(-)} = 4,3 ~ \text{мкс} ~~ \tau^{(+)} = 53,0 ~ \text{мкс}$}
	\end{center}

	\begin{center}
		\fbox{$\frac{\tau^{(+)}}{\tau^{(-)}} = 12,24$}
	\end{center}

	% =======================================================================================================================

	\section*{Задание 3.}

	\noindent Наибольший диапазон перестройки фазы реализуется на частоте $f_0 = 25$ кГц. При этом, границы этого диапазона
	$[-150, 73^o, -28,73^o]$.
	
	%------------------------------------------------------------------------------------------------------------------------

	\noindent Изучим частотную и фазовую характеристики двойного Т-образного моста. 
	\noindent Ширина полосы реженции $\Delta f = 41, 18 - 2,38 \approx 39 ~ \text{кГц} ~ \approx 4 F_0$.

	%------------------------------------------------------------------------------------------------------------------------

	\noindent Подключим ко входу источник прямоугольного импульса и проанализируем переходную характеристику. 
	Оценим время спада $\tau^{(-)} = 4,2$ мкс и нарастание $\tau^{(+)} = 67 - 12 = 55$ мкс.
	\noindent Теоретические значения

	\begin{equation*}
		\tau^{(\pm)} = \frac{1}{2 \pi f_0 \mu_{\pm}}, ~ \mu_{\pm} = 2 \pm \sqrt{3}
	\end{equation*}

	%------------------------------------------------------------------------------------------------------------------------

	\noindent Оценим частоты $f_0$ и добротность $Q = \frac{f_0}{\Delta f}$ нулей передачи.

	\begin{table}[h!]
		\begin{center}
			\begin{tabular}{|c|c|c|c|}
				\hline
				$R$, кОм & $f_0, кГц$ & $\Delta f$, Гц  & $Q$      \\ \hline
				4,9      & 9,94       & 100             & 99       \\ \hline
				5,0      & 10,00      & 1               & 10000    \\ \hline
				5,1      & 10,05      & 98              & 103      \\ \hline
			\end{tabular}
		\end{center}
	\end{table}

	\noindent Групповые задержки

	\begin{center}
		$R = 4,9$ кГц, $f = 10,05$ кГц | $\tau_g = 3$ мс, $\tau_{теор} = \frac{Q}{\pi f} = $ \\
		$R = 5,1$ кГц, $f = 9,95$  кГц | $\tau_g = 3$ мс, $\tau_{теор} = \frac{Q}{\pi f} = $
	\end{center}

	% ========================================================================================================================
	\section*{Задание 4.}

	\noindent Параметры компонентов.

	\begin{center}
		$L = 220$ мкГн \\
		$R = 91$  Ом   \\
		$C = 1,2$ нФ   \\
	\end{center}

	\noindent На макетной плате соберём схему полосового фильтра с указанными параметрами. 
	Подключив генератор синусоидального сигнала, измерим резонансную частоту $f_0$, коэффициент передачи $K_0$ и ширину
	$\Delta f$ пика по уровню $0,7 U_0$. Оценим добротность $Q = \frac{f_0}{\Delta f}$.
	
	\begin{center}
		$f_0 = 36,4$ кГц                   \\
		$K(f_0) = 1,1$                     \\
		$\Delta f = 40,4 - 32,6 = 7,7$ кГц \\
		$Q \approx 4,72$
	\end{center}

	% ------------------------------------------------------------------------------------------------------------------------

	\noindent Из тех же компонент соберём схемы фильтров верхних (ФВЧ) и нижних (ФНЧ) частот.

	\begin{enumerate}
		\item Для ФНЧ $\frac{K(f_0)}{K(0)} \approx 3,27$
		\item Для ФВЧ $\frac{K(f_0)}{K(\infty)} \approx 3,19$
	\end{enumerate}

	% ------------------------------------------------------------------------------------------------------------------------

	\noindent Подключив генератор прямоугольных импульсов, изучим переходные характеристики
	ФВЧ, ПФ, и ФНЧ. Прикинув по осциллограммам период колебаний и время их затухания
	до уровня $\frac{1}{e} = 0,37$, дадим оценку резонансной частоты $f_0$ и добротности $Q$.

	\begin{table}[h!]
		\begin{center}
			\begin{tabular}{|c|c|c|c|c|}
				\hline
						& $T$, мкс  & $\tau$, мкс  & $f_0$, кГц  & $Q$   \\ \hline
				ФНЧ 	& 2,4       & 2,8          & 366         & 7,1   \\ \hline
				ПФ  	& 3,3       & 2,6          & 392         & 4.9   \\ \hline
				ФВЧ 	& 2,8       & 2,8          & 366         & 6,1   \\ \hline
			\end{tabular}
		\end{center}
	\end{table}

	% ------------------------------------------------------------------------------------------------------------------------

	\begin{equation*}
		\tau_g = \frac{Q}{\pi f_0} \Rightarrow Q = \tau_g \pi f_0 = \frac{\rho}{r}
	\end{equation*}
	где $\rho = \sqrt{\frac{L}{C}}$. Отсюда по формулам находим расчетное значение добротности и торетическое значение грпповой задержки.

	\begin{table}[h!]
		\begin{center}
			\begin{tabular}{|c|c|c|c|c|}
				\hline
				$R$, Ом                     & 10    & 20   & 40   & 100  \\ \hline
				$\tau$, мс                  & 0,65  & 0,3  & 0,15 & 0,06 \\ \hline
				$\tau_{\text{теор}}$, мс    & 0,64  & 0,32 & 0,16 & 0,06 \\ \hline
				$Q$                         & 200   & 100  & 50   & 20   \\ \hline
			\end{tabular}
		\end{center}
	\end{table}

	% ------------------------------------------------------------------------------------------------------------------------

	\noindent Изучим графики распределения мощностей в резонансной
	LRC-цепи. Проверим выполнение закона суммирования мощностей на частоте резонанса
	и на границах полосы пропускания.

	\begin{center}
		$f_0 = 250$      кГц   \\
		$P_L = 175,545$  мВт   \\
		$P_R = 20,796$   мВт   \\
		$P_C = -177,606$ мВт   \\
		$\sum P_i = 18,69$ мВт \\
	\end{center}

	\noindent Границы полосы пропусканя $f_1 = 242$ кГц, $f_2 = 259$ кГц.

	\begin{center}
		$f_1 = 242$      кГц   \\
		$P_L = 85,06$  мВт   \\
		$P_C = -87,83$   мВт   \\
		$P_R = 9,36$ мВт   \\
		$\sum P_i = 6,59$ мВт \\
	\end{center}

	\begin{center}
		$f_2 = 259$      кГц   \\
		$P_L = 88,71$  мВт   \\
		$P_C = -89,54$   мВт   \\
		$P_R = 8,75$ мВт   \\
		$\sum P_i = 7,92$ мВт \\
	\end{center}

	% =======================================================================================================================

	\section{Задание 5.}

	Запишем параметры схемы: $f_0 = 100$ кГц, $\varrho = 570$ $\Rightarrow$ $\alpha = 0,057$, $\beta = 0,056$, $Q = 8,85$.

	% -----------------------------------------------------------------------------------------------------------------------

	\noindent Сопративление контура на резонансной частоте $R \approx 5$ кОм, полоса пропускания $\Delta f \approx 11,15$ кГц.

	% -----------------------------------------------------------------------------------------------------------------------

	Изучим зависимость частоты параллельного резонанса от $R$. Проверим формулу $f = f_0 \sqrt{1 - \beta^2}$, где $\beta = \frac{R}{\rho}$.

	\begin{table}[h!]
		\begin{center}
			\begin{tabular}{|c|c|c|c|c|}
				\hline
				$R$, Ом                  & 0,00   & 50,0 & 100,00  & 150,00  \\ \hline
				$f_{\text{эксп}}$, кГц   & 99,98  & 99,6 & 98,42   & 96,42   \\ \hline
				$f_{\text{теор}}$, кГц   & 100,00 & 99,0 & 98,40   & 96,60   \\ \hline
			\end{tabular}
		\end{center}
	\end{table}

	% ------------------------------------------------------------------------------------------------------------------------

	Фазовый сдвиг на частоте 2 кГц составляет $\frac{\pi}{4}$ при сопративлении $R = 11,5$ Ом.

	% ========================================================================================================================

	\section*{Задание 6.}

	\noindent Измерим частоты $f_p$, $f_0$ последовательного и параллельного резонансов по точкам пересечения нуля фазовой характеристикой.
	Получаем $f_0 = 100,5$ кГц, $f_p = 140$ кГц.
	Измерим полюсы $\Delta f_0$ и $\Delta f_p$, в которых фазовая характеристика изменяется в диапазоне $\pm 45$ в окрестностях
	резонансов. Получаем $\Delta f_p = 106,47 - 95,7 = 10,77$ кГц, $\Delta f_0 = 145,29 - 134,7 = 10,59$ кГц. 

	\noindent Расчитаем добротности $Q_p = \frac{f_p}{\Delta f_p} = 13,22$, $Q_p = \frac{f_0}{\Delta f_0} = 9,33$.
	Заметим, что $\frac{Q_p}{Q_0} \approx 1,41 \approx \sqrt{2}$.

\end{document}
