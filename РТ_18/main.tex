\input{preambule_article.tex}

\begin{document}
	\begin{center}
    
    \normalsize{Федеральное государственное автономное образовательное учреждение высшего образования}
    
    \textbf{НАЦИОНАЛЬНЫЙ ИССЛЕДОВАТЕЛЬСКИЙ УНИВЕРСИТЕТ \\ <<МОСКОВСКИЙ ФИЗИКО-ТЕХНИЧЕСКИЙ ИНСТИТУТ>>}
    \vspace{13ex}
    
    \textbf{Лабораторная работа 3.1.3\\Измерение магнитного поля Земли.}
    \vspace{40ex}
    
    \normalsize{Филиппенко Павел Сергеевич \\ студент группы Б01-001\\ 2 курс ФРКТ\\}
\end{center}
    
\vfill 
    
\begin{center}
г. Долгопрудный\\ 
2021 г.
\end{center}


\thispagestyle{empty} % выключаем отображение номера для этой страницы
\newpage
	
	\section*{Задание 1.}
	
	Соберем на макетной плате схема интегрирующей цепи с параметрами $R = 100$ Ом, $C = 1,05$ мкФ, тогда постоянная времени для этой цепи $\tau = R C = 105$ мкс.
	
	Экспериментально определим верхнюю граничную частоту $\nu_0$, подбирая $\nu_0$ таким образом, чтобы амплитуда выходного сигнала составила $70 \%$ от амплитуды входного сигнала. $U_{\text{вых}} = 0,7 U_{\text{вх}}$.
	
	Теоретическое значение $\nu_{\text{0теор}} = \frac{1}{2 \pi C R} = 1,516$ кГц. Экспериментальное значение $\nu_{\text{0экс}} = 1,51$ кГц.
	
	%---------------------------------------------------------------------------------------------------------------------------------------------------------------
	
	Снимем зависимость коэффициента передачи $K(\nu) = \frac{U_{\text{вых}}}{U_{\text{вх}}}$ от частоты $\nu = 2^n \nu_0$ в пределах $n \in [-2, 4]$. Построим график $K(\nu)$, а так же граф Боде -- $20 \lg K$ от$n = \log_2 (\nu / \nu_0)$.

	\begin{table}[h!]
		\begin{center}
			\begin{tabular}{|c|c|c|c|}
				\hline
				$n$& $\nu$, кГц & $K(\nu)$ &  $20 \lg K$      \\ \hline
				-2 & 0,377      & 0,96     & -0,35  \\ \hline
				-1 & 0,755      & 0,9      & -0,92  \\ \hline
				0  & 1,51       & 0,73     & -2,73  \\ \hline
				1  & 3,02       & 0,55     & -5,19  \\ \hline
				2  & 6          & 0,32     & -9,90  \\ \hline
				3  & 12         & 0,16     & -15,92 \\ \hline
				4  & 24,1       & 0,088    & -21,11 \\ \hline
			\end{tabular}
		\end{center}
		\caption{}
	\end{table}

	% график 1
	% график 2
	% картинка колебаний
	
	%---------------------------------------------------------------------------------------------------------------------------------------------------------------
	
	По осцилограмме прямоугольных сигналов оценим постоянную времени $\tau$,\\ измерив время нарастания фронта импульса от нуля до уровня $1 - 1/e \approx 0,63$.
	
	Экспериментально получено $\tau = 111$ мкс. Тогда $\nu_0 = \frac{1}{2 \pi \tau} \approx 1434$ Гц.
	
	% картинка прямоугольных сигналов. 
	
	%---------------------------------------------------------------------------------------------------------------------------------------------------------------
	
	Соберем на макетной плате схема интегрирующей цепи. Экспериментально определим нижнюю граничную частоту $\nu_0$.
	
	Теоретическое значение $\nu_{\text{0теор}} = \frac{1}{2 \pi C R} = 1,516$ кГц. Экспериментальное значение $\nu_{\text{0экс}} = 1,9$ кГц.
	
	%---------------------------------------------------------------------------------------------------------------------------------------------------------------
	
	Снимем зависимость коэффициента передачи $K(\nu) = \frac{U_{\text{вых}}}{U_{\text{вх}}}$ от частоты $\nu = 2^n \nu_0$ в пределах $n \in [-4, 2]$. Построим график $K(\nu)$, а так же граф Боде -- $20 \lg K$ от$n = \log_2 (\nu / \nu_0)$.
	
	
	\begin{table}[h!]
		\begin{center}
			\begin{tabular}{|c|c|c|c|}
				\hline
				n  & $\nu$, Гц & $K(\nu)$ & $20 \lg K$  \\ \hline
				-4 & 118,7     & 0,07     & -23,10 \\ \hline
				-3 & 237,5     & 0,11     & -19,17 \\ \hline
				-2 & 475       & 0,246    & -12,18 \\ \hline
				-1 & 950       & 0,43     & -7,33  \\ \hline
				0  & 1900      & 0,69     & -3,22  \\ \hline
				1  & 3800      & 0,87     & -1,21  \\ \hline
				2  & 7600      & 0,95     & -0,45  \\ \hline
			\end{tabular}
		\end{center}
		\caption{]
	\end{table}
	
	% график 1
	% график 2
	% картинка колебаний
	
	По осцилограмме прямоугольных сигналов оценим постоянную времени $\tau$,\\ измерив время спада вершины импульса от нуля до уровня $1/e \approx 0,37$.
	
	Экспериментально получено $\tau = 628,4$ мкс. Тогда $\nu_0 = \frac{1}{2 \pi \tau} \approx 1591$ Гц.
	
	% картинка прямоугольных сигналов. 

\end{document}