\documentclass[a4paper, 12pt]{article}

%%% Работа с русским языком
\usepackage{cmap}					% поиск в PDF
\usepackage{mathtext} 				% русские буквы в формулах
\usepackage[T2A]{fontenc}			% кодировка
\usepackage[utf8]{inputenc}			% кодировка исходного текста
\usepackage[russian]{babel}	% локализация и переносы

%%% Дополнительная работа с математикой
\usepackage{amsmath,amsfonts,amssymb,amsthm,mathtools} % AMS
\usepackage{icomma} % "Умная" запятая: $0,2$ --- число, $0, 2$ --- перечисление

%% Номера формул
%\mathtoolsset{showonlyrefs=true} % Показывать номера только у тех формул, на которые есть \eqref{} в тексте.

%% Шрифты
\usepackage{euscript}	 % Шрифт Евклид
\usepackage{mathrsfs} % Красивый матшрифт

%% Поля
\usepackage[left=2cm,right=2cm,top=2cm,bottom=2cm,bindingoffset=0cm]{geometry}

%% Русские списки
\usepackage{enumitem}
\makeatletter
\AddEnumerateCounter{\asbuk}{\russian@alph}{щ}
\makeatother

%%% Работа с картинками
\usepackage{graphicx}  % Для вставки рисунков
\graphicspath{{images/}{images2/}}  % папки с картинками
\setlength\fboxsep{3pt} % Отступ рамки \fbox{} от рисунка
\setlength\fboxrule{1pt} % Толщина линий рамки \fbox{}
\usepackage{wrapfig} % Обтекание рисунков и таблиц текстом

%%% Работа с таблицами
\usepackage{array,tabularx,tabulary,booktabs} % Дополнительная работа с таблицами
\usepackage{longtable}  % Длинные таблицы
\usepackage{multirow} % Слияние строк в таблице

%% Красная строка
\setlength{\parindent}{2em}

%% Интервалы
\linespread{1}
\usepackage{multirow}

%% TikZ
\usepackage{tikz}
\usetikzlibrary{graphs,graphs.standard}

%% Верхний колонтитул
\usepackage{fancyhdr}
\pagestyle{fancy}

%% Перенос знаков в формулах (по Львовскому)
\newcommand*{\hm}[1]{#1\nobreak\discretionary{}
	{\hbox{$\mathsurround=0pt #1$}}{}}

%% Мои дополнения
\usepackage{float} %Добавляет возможность работы с командой [H] которая улучшает расположение на странице
\usepackage{gensymb} %Красивые градусы
\usepackage{graphicx}               % Импорт изображений
\usepackage{caption} % Пакет для подписей к рисункам, в частности, для работы caption*

\begin{document}
    \begin{center}
    
    \normalsize{Федеральное государственное автономное образовательное учреждение высшего образования}
    
    \textbf{НАЦИОНАЛЬНЫЙ ИССЛЕДОВАТЕЛЬСКИЙ УНИВЕРСИТЕТ \\ <<МОСКОВСКИЙ ФИЗИКО-ТЕХНИЧЕСКИЙ ИНСТИТУТ>>}
    \vspace{13ex}
    
    \textbf{Лабораторная работа 3.1.3\\Измерение магнитного поля Земли.}
    \vspace{40ex}
    
    \normalsize{Филиппенко Павел Сергеевич \\ студент группы Б01-001\\ 2 курс ФРКТ\\}
\end{center}
    
\vfill 
    
\begin{center}
г. Долгопрудный\\ 
2021 г.
\end{center}


\thispagestyle{empty} % выключаем отображение номера для этой страницы
\newpage

    \section{Ход работы}

    %% подписать что есть что
    \begin{center}
        Запишем параметры установки \\
        $a = 25$ мм \\
        $h = 1.5$ мм
    \end{center}

    \begin{enumerate}
        \item Сняли зависимость $U(f)$ и $I(f)$ в области низких частот: 10-100 Гц для получаения зависимости
        амплитуды магнитного поля внутри экрана от частоты $\xi_{0c}(f)$. Результаты измерений приведены в таблице %какой.

        \begin{equation}
            \xi_{0c} = \frac{U}{f I}
        \end{equation}

        \item Одновременно исследовали зависимоть $\xi_{0c}$ и фазового сдвига $\Delta \psi$ от частоты в диапазоне высоких частот
        0.1-35 кГц. Результаты измерений приведены в таблице %какой.
        \\
        \textbf{Замечание:} сдвиг фазы $\Delta \varphi$, измеренный по экрану осцилографа, будет отличаться от
        фазового сдвига между магнитными полями вне и внутри экрана на $\pi/2$.

        \[ \Delta \psi = \Delta \varphi + \frac{\pi}{2} \]
    \end{enumerate}

    \begin{table}[]
    \centering
    \begin{tabular}{|c|c|c|c|}
    \hline
    $f$, Гц & $I$, мА & $U$, мВ  & $\xi_{0c} \cdot 10^{-3}$ \\ \hline
    10      & 449,80  & 86,3     & 19,19                    \\ \hline
    20      & 452,47  & 172,0    & 19,01                    \\ \hline
    30      & 449,65  & 254,2    & 18,84                    \\ \hline
    40      & 445,28  & 331,7    & 18,62                    \\ \hline
    50      & 439,86  & 403,5    & 18,35                    \\ \hline
    60      & 433,75  & 469,3    & 18,03                    \\ \hline
    70      & 427,23  & 528,8    & 17,68                    \\ \hline
    80      & 420,61  & 582,0    & 17,30                    \\ \hline
    90      & 413,93  & 629,4    & 16,89                    \\ \hline
    100     & 407,56  & 671,3    & 16,47                    \\ \hline
    \end{tabular}
    \caption{Данные в диапазоне низких частот}
    \label{table_low_f}
\end{table}

    \begin{table}[h!]
    \centering
    \begin{tabular}{|c|c|c|c|c|c|}
    \hline
    $f$, Гц & $I$, мА  & $U$, мВ   & $\xi_{0c} \cdot 10^{-6}$  & $\Delta \varphi$ & $\Delta \psi$ \\ \hline
    100     & 407,56   & 671,3     & 16 471,19                 & -1,00            & 0,57          \\ \hline
    500     & 316,84   & 966,0     & 6 097,71                  & -0,23            & 1,34          \\ \hline
    1000    & 279,48   & 884,0     & 3 163,02                  & 0,00             & 1,57          \\ \hline
    2000    & 209,62   & 662,2     & 1 579,52                  & 0,20             & 1,77          \\ \hline
    3000    & 159,35   & 496,8     & 1 039,22                  & 0,35             & 1,92          \\ \hline
    4000    & 126,14   & 385,7     & 764,43                    & 0,49             & 2,06          \\ \hline
    5000    & 103,39   & 310,4     & 600,44                    & 0,64             & 2,21          \\ \hline
    6000    & 87,14    & 253,5     & 484,85                    & 0,81             & 2,38          \\ \hline
    7000    & 74,99    & 211,7     & 403,29                    & 0,90             & 2,47          \\ \hline
    8000    & 65,48    & 180,4     & 344,38                    & 0,94             & 2,51          \\ \hline
    9000    & 57,86    & 153,8     & 295,35                    & 1,09             & 2,66          \\ \hline
    10000   & 51,56    & 133,8     & 259,50                    & 1,22             & 2,79          \\ \hline
    12000   & 41,22    & 10,1      & 20,46                     & 1,57             & 3,14          \\ \hline
    14000   & 33,90    & 80,0      & 168,55                    & 1,79             & 3,36          \\ \hline
    15000   & 30,82    & 71,7      & 155,09                    & 1,89             & 3,46          \\ \hline
    16000   & 28,03    & 64,6      & 144,07                    & 1,90             & 3,47          \\ \hline
    17000   & 25,48    & 58,8      & 135,73                    & 1,92             & 3,49          \\ \hline
    18000   & 23,14    & 53,5      & 128,45                    & 2,04             & 3,61          \\ \hline
    19000   & 20,96    & 49,1      & 123,27                    & 2,15             & 3,72          \\ \hline
    20000   & 18,93    & 45,2      & 119,36                    & 2,18             & 3,75          \\ \hline
    21000   & 17,03    & 42,0      & 117,45                    & 2,24             & 3,81          \\ \hline
    22000   & 15,23    & 39,1      & 116,70                    & 2,34             & 3,91          \\ \hline
    23000   & 13,52    & 36,6      & 117,67                    & 2,47             & 4,04          \\ \hline
    24000   & 11,90    & 34,3      & 120,12                    & 2,52             & 4,09          \\ \hline
    25000   & 10,34    & 32,4      & 125,29                    & 2,60             & 4,17          \\ \hline
    26000   & 8,85     & 30,6      & 132,93                    & 2,65             & 4,22          \\ \hline
    27000   & 7,42     & 29,0      & 144,70                    & 2,72             & 4,29          \\ \hline
    28000   & 6,10     & 27,4      & 160,42                    & 2,77             & 4,34          \\ \hline
    29000   & 4,83     & 26,0      & 185,51                    & 2,87             & 4,44          \\ \hline
    30000   & 3,60     & 24,0      & 222,22                    & 2,99             & 4,56          \\ \hline
    31000   & 2,76     & 22,0      & 257,22                    & 3,14             & 4,71          \\ \hline
    32000   & 2,38     & 20,8      & 273,11                    & 3,33             & 4,90          \\ \hline
    33000   & 2,70     & 18,9      & 212,51                    & 3,67             & 5,24          \\ \hline
    34000   & 3,51     & 17,0      & 142,26                    & 4,04             & 5,61          \\ \hline
    35000   & 4,57     & 15,0      & 93,78                     & 4,71             & 6,28          \\ \hline
    \end{tabular}
    \caption{Данные в диапазоне высоких частот}
    \label{table_hight_f}
\end{table}

    \section{Обработка результатов}

    В области низких частот построим график $\xi_{0c}(f^2)$.

    \begin{figure}
        \centering
        \includegraphics[width = \textwidth]{xi0c(1).pdf}
        \caption{График зависимости $\xi_{0c}$ от $f^2$}
        \label{xi0c(1)}
    \end{figure}

    Экстраполируя прямую к $f = 0$ найдемамплитуду внешнего поля $\xi_0$.

    \begin{center}
        \fbox{$\xi_0 = (0.019 ~ \pm ~ 0.000037)$}
    \end{center}

    Изобразим зависимость частоты фазового сдвига $\Delta \psi$ от $\sqrt{f}$ в диапазоне высоких частот.
    Проведем наилучшую прямую, заметим, что при $f = 0$ значение ординаты $y_0 = 0.75 \approx \pi/4$.

    \begin{figure}
        \centering
        \includegraphics[width = \textwidth]{DeltaPsi.pdf}
        \caption{График зависимости $\Delta \psi$ от $\sqrt{f}$}
        \label{DeltaPsi}
    \end{figure}

    По наклону прямой найдем значение проводимости материала экрана.

    \begin{equation}
        \Delta \psi = \frac{\pi}{4} + \frac{h \sqrt{2 \pi \mu_0 \sigma f}}{\sqrt{2}}
    \end{equation}

    \begin{center}
        \fbox{$\sigma = (5.178 \cdot 10^7 ~ \pm ~ )$ См/м} \\
        Табличное значение проводимости меди $\sigma_{\text{табл}} = 6.48 \cdot 10^7$ См/м.
    \end{center}

    \textbf{Замечание:} обратим внимание на то, что полученное значение проводимости меди меньше табличного. %% Почему
    \textbf{Замечание:} обратим внимание, кроме того, на резкий рост значений графика в области частот, начиная с 30-31 кГц. %% Почему

    Импользуя ранее найденное значение амплитуды внешнего поля $\xi_0$ и результаты измерений $\xi_{0c}$
    в области высоких частот, посчитаем коэффициент ослабления магнитного поля в соответствии с формулой.

    \begin{equation}
        \frac{|H_{0c}|}{|H_{0}|} = \frac{\xi_{0c}(f)}{\xi_0} = \frac{U}{f I \xi_0}
    \end{equation}

    Построим график зависимости $\frac{|H_{0c}|}{|H_{0}|}$ от $\sqrt(f)$.

    \begin{figure}
        \centering
        \includegraphics[width = \textwidth]{xi_xi0(sqrt).pdf}
        \caption{График зависимости $\frac{|H_{0c}|}{|H_{0}|}$ от $\sqrt(f)$}
        \label{xi_xi0(sqrt)}
    \end{figure}

    Расчитаем аналогичную теоретическую зависимость по формуле

    \begin{equation}
        H_{0c} = \frac{2 H_0}{a k \sinh{k h} + 2 \cosh{k h}}
    \end{equation}

    где
    \begin{equation}
        \delta = \sqrt{\frac{1}{\pi \mu_0 \mu \sigma f}}
    \end{equation}
    глубина проникновения, а
    \begin{equation}
       k = \frac{1 + i}{\delta} 
    \end{equation}

    Стоит воспользоваться приближением при низких и высоких частотах. 
    При низких частотах выражение приближается формулой 
    \begin{equation}
        \frac{|H_1|}{|H_0|} = \frac{1}{\sqrt{1 + \frac{1}{4} (2 \pi a h \sigma \mu_0 f)^2}}
    \end{equation}

    При высоких частотах 
    \begin{equation}
        \frac{|H_1|}{|H_0|} = \frac{2 \sqrt{2} \delta}{a} e^{-\frac{h}{\delta}}
    \end{equation}

    Полученная таким образом теоретическая зависимость достаточно хорошо ложиться на эксперементальные данные.

    \begin{figure}
        \centering
        \includegraphics[width = \textwidth]{exp_th(sqrt).pdf}
        \caption{Сравнение экспериментальной зависимости с теоретической}
        \label{exp_th(sqrt)}
    \end{figure}

    Воспользовавшись найденным значением $\sigma$, вычислим глубину проникновения $\delta$ при частотах 50 Гц и $10^5$ Гц.

    \begin{center}
        На частоте 50 Гц глубина проникновения $\delta = ( ~ \pm ~ )$ \\
        На частоте $10^5$ Гц глубина проникновения $\delta = ( ~ \pm ~ )$
    \end{center}
    
\end{document}