\input{preambule_article}

\begin{document}
    \begin{center}
    
    \normalsize{Федеральное государственное автономное образовательное учреждение высшего образования}
    
    \textbf{НАЦИОНАЛЬНЫЙ ИССЛЕДОВАТЕЛЬСКИЙ УНИВЕРСИТЕТ \\ <<МОСКОВСКИЙ ФИЗИКО-ТЕХНИЧЕСКИЙ ИНСТИТУТ>>}
    \vspace{13ex}
    
    \textbf{Лабораторная работа 3.1.3\\Измерение магнитного поля Земли.}
    \vspace{40ex}
    
    \normalsize{Филиппенко Павел Сергеевич \\ студент группы Б01-001\\ 2 курс ФРКТ\\}
\end{center}
    
\vfill 
    
\begin{center}
г. Долгопрудный\\ 
2021 г.
\end{center}


\thispagestyle{empty} % выключаем отображение номера для этой страницы
\newpage

    \section*{Обработка эксперементальных данных.}

    \noindent Проведем градуирование магнита.
    \begin{table}[h!]
    \begin{center}
        \begin{tabular}{|c|c|}
            \hline
            $B$, мТл & $I$, А \\ \hline
            1057     & 2,0   \\ \hline
            1032     & 1,8  \\ \hline
            978      & 1,6  \\ \hline
            935      & 1,4  \\ \hline
            845      & 1,2  \\ \hline
            730      & 1,0   \\ \hline
            611      & 0,8  \\ \hline
            491      & 0,6  \\ \hline
            338      & 0,4  \\ \hline
            \end{tabular}
    \end{center}
    \caption{}
\end{table}
    % график градуирования

    \noindent Измерим вольт-амперную характеристику образца.
    \begin{table}[h!]
    \begin{center}
        \begin{tabular}{|c|c|}
            \hline
            $I$, мА & $U$, мкВ \\ \hline
            0,2     & 361      \\ \hline
            0,3     & 530      \\ \hline
            0,4     & 703      \\ \hline
            0,5     & 875      \\ \hline
            0,6     & 1043     \\ \hline
            0,7     & 1220     \\ \hline
            0,8     & 1392     \\ \hline
            0,9     & 1565     \\ \hline
            1,0     & 1743     \\ \hline
        \end{tabular}
    \end{center}
    \caption{Вольт-амперная характиристика образца}
\end{table}
    
    \begin{figure}[h!]
        \centering
        \includegraphics[scale = 1]{VAC.jpg}
        \caption{}
    \end{figure}

    \noindent Параметры образца.
    \begin{center}
        $a = 2,2$ мм -- ширина образца              \\
        $h = 2,5$ мм -- толщина образца             \\
        $L = 3,0$ мм -- расстояние между контактами \\
    \end{center}

    \noindent Удельное сопротивление образца можем посчитать по формуле
    \begin{equation*}
        \rho_0 = \frac{U_{35} a h}{I L}
    \end{equation*}
    \noindent Величину $\frac{U_{35}}{I}$ найдем из графика.

    \begin{center}
        \fbox{$\rho_0 = 0,312 ~ \text{Ом} \cdot \text{см}$}
    \end{center}

    \noindent Найдем отсюда удельную проводимость.

    \begin{center}
        \fbox{$\sigma = 3,2 ~ (\text{Ом} \cdot \text{см})^{-1}$}
    \end{center}

    \noindent Снимем зависимость ЭДС Холла от значения индукции магнитного поля при разных значениях продольного тока.
    Заметим, что напряжение на контактах связано не только с эффектом Холла, но и с 
    оммическим падением напряжения вдоль пластины. Исключить этот эффект можно двумя способами:

    \begin{enumerate}
        \item Изменять направление магнитного поля, пронизывающего образец. При обращении поля знак
        ЭДС Холла меняется, поэтому ЭДС Холла $U_{34}$ может быть определена по формуле
        \begin{equation*}
            U_{\bot} = \frac{U^{(+)} - U^{(-)}}{2}
        \end{equation*}
        \item Можно исключить влияние оммического падения напряжения, измеряя падение напряжения на образце $U_0$
        в отсутсвии магнитного поля. Тогда ЭДС Холла вычисляться по формуле
        \begin{equation*}
            U_{\bot} = U_{34} - U_0
        \end{equation*}
    \end{enumerate}

    \newpage
    % ------------------------------------------------------------------------------------------------------------------------
    \begin{center}
        $I = 1$    мА  \\
        $U_0 = -38$ мкВ \\
    \end{center}
    \begin{table}[h!]
    \begin{center}
        \begin{tabular}{|c|c|c|c|c|}
            \hline
            $U^{(+)}$, мкВ & $U^{(-)}$, мкВ & $U_{\bot}$, мкВ & $B$, мкТл & $I$, А \\ \hline
            166            & -237           & 202             & 1057      & 2,0    \\ \hline
            157            & -230           & 194             & 1032      & 1,8    \\ \hline
            149            & -221           & 184             & 978       & 1,6    \\ \hline
            137            & -210           & 174             & 935       & 1,4    \\ \hline
            121            & -193           & 157             & 845       & 1,2    \\ \hline
            99             & -172           & 136             & 730       & 1,0    \\ \hline
            75             & -148           & 112             & 611       & 0,8    \\ \hline
            49             & -122           & 86              & 491       & 0,6    \\ \hline
            23             & -95            & 59              & 338       & 0,4    \\ \hline
        \end{tabular}
    \end{center}
    \caption{}
\end{table}
    \begin{figure}[h!]
        \centering
        \includegraphics[scale = 1]{HallEffectI1.jpg}
        \caption{}
    \end{figure}
    % ------------------------------------------------------------------------------------------------------------------------
    \newpage
    \begin{center}
        $I = 0,5$    мА  \\
        $U_0 = -17$ мкВ \\
    \end{center}
    \begin{table}[h!]
    \begin{center}
        \begin{tabular}{|c|c|c|c|c|}
            \hline
            $U^{(+)}$, мкВ & $U^{(-)}$, мкВ & $U_{\bot}$, мкВ & $B$, мкТл & $I$, А \\ \hline
            52             & -72            & 62              & 1057      & 2      \\ \hline
            50             & -69            & 60              & 1032      & 1,8    \\ \hline
            47             & -66            & 57              & 978       & 1,6    \\ \hline
            44             & -63            & 54              & 935       & 1,4    \\ \hline
            39             & -58            & 49              & 845       & 1,2    \\ \hline
            32             & -52            & 42              & 730       & 1      \\ \hline
            25             & -44            & 35              & 611       & 0,8    \\ \hline
            17             & -36            & 27              & 491       & 0,6    \\ \hline
            8              & -28            & 18              & 338       & 0,4    \\ \hline
            \end{tabular}
    \end{center}
\end{table}
    \begin{figure}[h!]
        \centering
        \includegraphics[scale = 1]{HallEffectI2.jpg}
        \caption{}
    \end{figure}
    % ------------------------------------------------------------------------------------------------------------------------
    \newpage
    \begin{center}
        $I = 0,3$    мА  \\
        $U_0 = -10$ мкВ \\
    \end{center}
    \begin{table}[h!]
    \begin{center}
        \begin{tabular}{|c|c|c|c|c|}
            \hline
            $U^{(+)}$, мкВ & $U^{(-)}$, мкВ & $U_{\bot}$, мкВ & $B$, мкТл & $I$, А \\ \hline
            85             & -120           & 103             & 1057      & 2      \\ \hline
            82             & -115           & 99              & 1032      & 1,8    \\ \hline
            77             & -110           & 94              & 978       & 1,6    \\ \hline
            71             & -105           & 88              & 935       & 1,4    \\ \hline
            63             & -96            & 80              & 845       & 1,2    \\ \hline
            52             & -85            & 69              & 730       & 1      \\ \hline
            40             & -73            & 57              & 611       & 0,8    \\ \hline
            26             & -60            & 43              & 491       & 0,6    \\ \hline
            13             & -47            & 30              & 338       & 0,4    \\ \hline
        \end{tabular}
    \end{center}
    \caption{}
\end{table}
    \begin{figure}[h!]
        \centering
        \includegraphics[scale = 1]{HallEffectI3.jpg}
        \caption{}
    \end{figure}
    % ------------------------------------------------------------------------------------------------------------------------
    \noindent По полученным данным вычислим концентрацию носителей зарядов в образце $n$.

    \begin{equation*}
        \varepsilon_h = \frac{I B}{n e a} = R_h \frac{I B}{a}
    \end{equation*}

    \noindent Найдем подвижность носителей зарядов $\mu$ в образце, спользуя формулу
    \begin{equation*}
        \sigma = e n \mu \Rightarrow \mu = \frac{\sigma}{e n}
    \end{equation*}

    \noindent где $e$ -- элементарный заряд, а $R_h = \frac{1}{n e}$ -- постоянная Холла. Тогда

    \begin{table}[h!]
    \begin{center}
        \begin{tabular}{|c|c|c|c|}
            \hline
            $I$  мА & $R_h \cdot 10^{-3}  ~ \frac{\text{Ом}}{\text{А}}$ & $n \cdot 10^{16} ~ \text{см}^{-3}$ & $\mu \cdot 10^3 ~ \frac{\text{см}^2}{\text{В} \cdot \text{с}}$  \\ \hline
            1,0     & 0,418                                             & 1,50                               & 1,33                                                            \\ \hline
            0,5     & 0,440                                             & 1,40                               & 1,43                                                            \\ \hline
            0,3     & 0,447                                             & 1,39                               & 1,44                                                            \\ \hline                
        \end{tabular}
    \end{center}
    \caption{}
\end{table}

    \noindent \textbf{Вывод:} таким образом, в работе мы нашли концентрацию
    и подвижность носителей зарядов в образце.\\
    Отличие полученных величин от табличных свидетельствует о том,
    что изучаемом образце присутствовали примеси.

\end{document}