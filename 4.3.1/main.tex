\documentclass[a4paper, 12pt]{article}

%%% Работа с русским языком
\usepackage{cmap}					% поиск в PDF
\usepackage{mathtext} 				% русские буквы в формулах
\usepackage[T2A]{fontenc}			% кодировка
\usepackage[utf8]{inputenc}			% кодировка исходного текста
\usepackage[russian]{babel}	% локализация и переносы

%%% Дополнительная работа с математикой
\usepackage{amsmath,amsfonts,amssymb,amsthm,mathtools} % AMS
\usepackage{icomma} % "Умная" запятая: $0,2$ --- число, $0, 2$ --- перечисление

%% Номера формул
%\mathtoolsset{showonlyrefs=true} % Показывать номера только у тех формул, на которые есть \eqref{} в тексте.

%% Шрифты
\usepackage{euscript}	 % Шрифт Евклид
\usepackage{mathrsfs} % Красивый матшрифт

%% Поля
\usepackage[left=2cm,right=2cm,top=2cm,bottom=2cm,bindingoffset=0cm]{geometry}

%% Русские списки
\usepackage{enumitem}
\makeatletter
\AddEnumerateCounter{\asbuk}{\russian@alph}{щ}
\makeatother

%%% Работа с картинками
\usepackage{graphicx}  % Для вставки рисунков
\graphicspath{{images/}{images2/}}  % папки с картинками
\setlength\fboxsep{3pt} % Отступ рамки \fbox{} от рисунка
\setlength\fboxrule{1pt} % Толщина линий рамки \fbox{}
\usepackage{wrapfig} % Обтекание рисунков и таблиц текстом

%%% Работа с таблицами
\usepackage{array,tabularx,tabulary,booktabs} % Дополнительная работа с таблицами
\usepackage{longtable}  % Длинные таблицы
\usepackage{multirow} % Слияние строк в таблице

%% Красная строка
\setlength{\parindent}{2em}

%% Интервалы
\linespread{1}
\usepackage{multirow}

%% TikZ
\usepackage{tikz}
\usetikzlibrary{graphs,graphs.standard}

%% Верхний колонтитул
\usepackage{fancyhdr}
\pagestyle{fancy}

%% Перенос знаков в формулах (по Львовскому)
\newcommand*{\hm}[1]{#1\nobreak\discretionary{}
	{\hbox{$\mathsurround=0pt #1$}}{}}

%% Мои дополнения
\usepackage{float} %Добавляет возможность работы с командой [H] которая улучшает расположение на странице
\usepackage{gensymb} %Красивые градусы
\usepackage{graphicx}               % Импорт изображений
\usepackage{caption} % Пакет для подписей к рисункам, в частности, для работы caption*

\begin{document}
\begin{center}
    
    \normalsize{Федеральное государственное автономное образовательное учреждение высшего образования}
    
    \textbf{НАЦИОНАЛЬНЫЙ ИССЛЕДОВАТЕЛЬСКИЙ УНИВЕРСИТЕТ \\ <<МОСКОВСКИЙ ФИЗИКО-ТЕХНИЧЕСКИЙ ИНСТИТУТ>>}
    \vspace{13ex}
    
    \textbf{Лабораторная работа 3.1.3\\Измерение магнитного поля Земли.}
    \vspace{40ex}
    
    \normalsize{Филиппенко Павел Сергеевич \\ студент группы Б01-001\\ 2 курс ФРКТ\\}
\end{center}
    
\vfill 
    
\begin{center}
г. Долгопрудный\\ 
2021 г.
\end{center}


\thispagestyle{empty} % выключаем отображение номера для этой страницы
\newpage

\noindent \textbf{Цель работы}
Исследовать явления дифракции Френеля и Фраунгофера на щели, изучить влияние дифракции на разрешающую способность оптических инструментов. \\
\textbf{Приборы и материалы}
Оптическая скамья, ртутная лампа, монохроматор, щели с регулируемой шириной, рамка с вертикальной нитью, двойная щель, микроскоп на поперечных салазках с микрометрическим винтом.

\section*{Описание работы}

\subsection*{А. Дифракция Френеля}
\begin{figure}[h]
	\includegraphics[scale=0.7]{0.png}
	\centering
	\caption{Схема установки 1.}
\end{figure}
Схема установки представлена на Рис. 1. Световые лучи освещают щель $S_2$ и испытывают на ней дифракцию. Дифракционная картина рассматривается с помощью микроскопа М, сфокусированного на некоторую плоскость наблюдения П. Щель $S_2$ освещается параллельным пучком монохроматического света с помощью коллиматора, образованного объективом $O_1$ и щелью $S_1$, находящейся в его фокусе. На $S_1$ сфокусированно изображение спектральной линии, выделенной из спектра ртутной лампы Л при помощи монохроматор $C$, в котором используется призма прямого зрения. \\
Распределение интенсивности света в плоскости П рассчитаем с помощью зон Френеля. При освещении $S_2$ параллельным пучком лучей (плоская зона) зоны Френеля представляют собой плоскости, параллельные краям щели. Результирующая амплитуда в точке наблюдения определеяется суперпозицией колебаний от тех зон Френеля, которые не перекрыты створками щели. Графическое определение результирующей амплитуды производится с помощью векторной диаграммы -- спирали Корню. Суммарная ширина $m$ зон Френеля $z_m$ определяется соотношение
\begin{equation}
	z_m = \sqrt{am\lambda},
\end{equation}
где $a$ -- расстояние от щели до плоскости П. Вид наблюдаемой картины определяется \textit{числом Френеля} $\Phi$:
$$
	\Phi^2 = \dfrac{D}{\sqrt{a\lambda}}
$$
-- число зон Френеля, которые укладываются в ширине щели $D$. $p = \frac{1}{\Phi^2}$ называется \textit{волновым параметром}. Дифракционной картины нет, когда П совпадает с плоскостью щели. При малом удалении от щели $\Phi \gg 1$ и картина наблюдается в узкой убласти на границе света и тени у краёв экрана. При последующих удалениях две группы дифракционных полос перемещаются независимо и каждая образует картину дифракции Френеля на экране. Распределение интенсивности может быть найдено с помощью спирали Корню. При дальнейшем увеличении $a$ две системы полос сближаются и накладываются друг на друга, распределение интенсивности определяется числом зон Френеля в полуширине щели. Если их $m$, то будет набюдаться $m-1$ тёмная полоса.
\subsection*{Б. Дифракция Фраунгофера на щели}
\begin{wrapfigure}{r}{0.5\textwidth}
	\begin{center}
		\includegraphics[width = 0.4\textwidth]{2.png}
	\end{center}
	\caption{Построение зон Френеля}
\end{wrapfigure}
Для выкладок ниже нам потребуется знать \textit{принцип Гюйгенса-Френеля}. Он формулируется следующим образом:\\
\textit{Каждый элемент волнового фронта можно рассматривать как центр  вторичного возмущения, порождающего вторичные сферические волны, а результирующее световое поле  в каждой точке пространства будет определяться интерференцией этих волн.}\\
Теперь рассмотрим первое применение этого принципа, получившее название \textit{метод зон Френеля}

\begin{wrapfigure}{r}{0.3\textwidth}
	\begin{center}
		\includegraphics[width = 0.3\textwidth]{1.png}
	\end{center}
	\caption{К фазовым соотношениям при дифракции Фраунгофера}
	\vspace{+30pt}
\end{wrapfigure}

Для этого рассмотрим действие световой волны действующей из точки $A$ в какой-то точке $B$.
В этом случае можно, взяв точку $M_0$ в качестве центра (см. рис. 1), построить ряд концентрических сфер, радиусы которых начинаются с $b$ и увеличиваются каждый раз на половину длины волны $\frac{\lambda}{2}$. При пересечении с плоским фронтом волны $F$ эти сферы дадут концентрические окружности. Таким образом, на фронте волны появятся кольцевые зоны (зоны Френеля) с радиусами $r_1, r_2$ и т. д.

Из геометрических соображений посчитав, можно получить, что
\begin{equation}
	r_i = i \sqrt{a \lambda}
\end{equation}

Картина дифракции упрощается, когда ширина щели становится значительно меньше ширины первой зоны Френеля, т.е. если
\begin{equation}
	D \ll\sqrt{a \lambda}
\end{equation}
Это условие всегда выполняется при достаточно большом $a$. В этом случае говорят, что \textit{дифракция Фраунгофера}. Дифракционную картину в этом случае называются \textit{дифракцией Фраунгофера}. При выполнении пункта $(2)$ у нас упрощаются фазовые соотношения, что поясняет рис. 2, в итоге с хорошим приближением можно считать, что разность хода между крайними лучами, приходящими от щели в точке наблюдения $P$, с хорошим приближением равна
\begin{equation}
	\Delta = r_2 - r_1 \approx D \sin \theta \approx D \cdot \theta
\end{equation}
Здесь предполагается, что $\theta$ достаточно мал.
Дифракцию Фраунгофера можно наблюдать на установке Рис. 1, но для удобства к подобной установке добавляется объектив $O_2$.

\begin{figure}[h]
	\includegraphics[width = 0.7\textwidth]{3.png}
	\centering
	\caption{Схема установки 2.}
\end{figure}
Дифракционная картина здесь наблюдается в фокальной плоскости объектива $O_2$. Каждому значению $\theta$ соответствует в этой плоскости точка, отстоящая от оптической оси на расстоянии
\begin{equation}
	X = f_2 \tan \theta \approx f_2 \theta.
\end{equation}
Объектив не вносит разности хода между интерферирующими лучам, поэтому в его фокальной плоскости наблюдается неискажённая дифракционная картина. При $\theta = 0$ разность хода между лучами нулевая, поэтому в центре поля зрения дифракционный максимум. Первый минимум соответствует $\theta_1$ такому, что в точке наблюдения разность хода пробегаем все значения от 0 до $2\pi$. Аналогично рассуждая, для $m$-й полосы
\begin{equation}
	\theta_m = \frac{m \lambda}{D}
\end{equation}
Расстояние $X_m$ тёмной полосы от оптической оси из (5) и (6)
\begin{equation}
	X_m = f_2m\frac{\lambda}{D}
\end{equation}
\subsection*{В. Дифракция Фраунгофера для двух щелей}
Для наблюдения дифракции Фраунгофера на двух щелях $S_2$ заменим экраном Э с двумя щелями. При этом для оценки влияния ширины входной щели на чёткость вместо $S_1$ поставим щель с микрометрическим винтом.
\begin{figure}[h]
	\includegraphics[width = 0.7\textwidth]{4.png}
	\centering
	\caption{Схема установки 3.}
\end{figure}
Два дифракционных изображения входной щели, одно из которых образовано лучами, прошедшими через левую, а другое -- через правую щели, накладываются друг на друга.
Если входная щель достаточно узка, то дифракционная картина в плоскости П подобна той, что получалась при дифракции на одной щели, однако вся картинка испещерена рядом дополнительных узких полос, наличие которых объясняется суперпозицией световых волн через разные щели. Светлая интерфереционная полоса наблюдается в случаях, когда разность хода равна целому числу длин волн. Таким образом, угловая координата максимума порядка $m$ равна
\begin{equation}
	\theta_m = \dfrac{m \lambda}{d},
\end{equation}
где $d$ -- расстояние между щелями. Отсюда расстояние между соседними интерфереционными полосами в плоскости П равно
\begin{equation}
	\delta x = f_2 \dfrac{\lambda}{d}
\end{equation}
Число интерференционных полос укладывающихся в области центрального максимума равна отношению ширины главного максимума $\frac{2\lambda f_2}{D}$ к расстоянию между соседними полосами:
\begin{equation}
	n = \dfrac{2\lambda f_2}{D} \dfrac{1}{\delta f}= \dfrac{2d}{D}.
\end{equation}
При дифракции света на двух щелях чёткая система интерференционных полос наблюдается только при достаточно узкой ширине входной щели $S$. При увеличении ширины картинка пропадает и появляется вновь, но полосы при этом сильно размыты и видны плохо.
\subsection*{Г. Влияние дифракции на разрешающую способность оптического инструмента}
\begin{figure}[h]
	\includegraphics[width = 0.8\textwidth]{5.png}
	\centering
	\caption{Схема установки 4.}
\end{figure}
В отсутствие щели $S_2$ линзы $O_1$ и $O_2$ создают на плоскости П изоюражение щели $S_1$ и это изображение рассматриваются микроскопом М. Таким образом, установку можно рассматривать как оптический инструмент, предназначенные для получения изображения предмета. Если перед $O_2$ расположить $S_2$, то изображение объекта будет искажено из-за дифракции. Чем меньше ширина щели, тем сильнее искажение. Качественной характеристикой этого искажения может служить $\varphi_{min}$ --- минимальное угловое расстояние между объектами (источниками), которые всё ещё воспринимаются как раздельные. Поместим вместо $S_1$ экран Э с двумя щелями с расстоянием $d$. Тогда на $S_2$ будут падать два пучка света с углом
\begin{equation}
	\varphi = \dfrac{d}{f_1}
\end{equation}
Из геометрии расстояние $l$ между изображениями щелей в плоскости П равно
\begin{equation}
	l = \varphi f_2 = d \dfrac{f_2}{f_1}.
\end{equation}
Ширина $\Delta \varphi$ определяется дифракцией на $S_2$. Условия, при которых изображения различимы разные для разных наблюдателей, поэтому используют \textit{критерий Рэлея} -- \textit{максимум одного дифракционного пятна должен совпадать с минимумом другого}. В наших условиях это значит, что угловая полуширина $\frac{\lambda}{D}$ равна угловому расстоянию $\frac{l}{f_2}$.
\end{document}
