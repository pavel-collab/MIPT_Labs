\input{preambule_article}

\begin{document}
    % \begin{center}
    
    \normalsize{Федеральное государственное автономное образовательное учреждение высшего образования}
    
    \textbf{НАЦИОНАЛЬНЫЙ ИССЛЕДОВАТЕЛЬСКИЙ УНИВЕРСИТЕТ \\ <<МОСКОВСКИЙ ФИЗИКО-ТЕХНИЧЕСКИЙ ИНСТИТУТ>>}
    \vspace{13ex}
    
    \textbf{Лабораторная работа 3.1.3\\Измерение магнитного поля Земли.}
    \vspace{40ex}
    
    \normalsize{Филиппенко Павел Сергеевич \\ студент группы Б01-001\\ 2 курс ФРКТ\\}
\end{center}
    
\vfill 
    
\begin{center}
г. Долгопрудный\\ 
2021 г.
\end{center}


\thispagestyle{empty} % выключаем отображение номера для этой страницы
\newpage

    Измерим параметры магнитных шариков: $m_1 = 820$ г, $m_2 = 815$ г, $d = 5,9$ мм.\\
    Величину магнитного момента двух одинаковых шариков можно расчитать, зная их массу и определив
    максимальное расстояние $r_{max}$, на котором они удерживают друг друга.

    \begin{equation*}
        P_m = \sqrt{\frac{4 \pi mg r_{max}^4}{6 \mu_0}}
    \end{equation*}

    Перепишем выражение с учетом $\mu_0 = 4 \pi \cdot 10^{-7}$

    \begin{equation*}
        P_m = \sqrt{\frac{mg r_{max}^4}{6} \cdot 10^7}
    \end{equation*}

    В процессе эксперемента получено $r_{max} = 23$ мм.

    \begin{center}
        \fbox{$P_m = 61,84 ~ \text{А} \cdot \text{м}^2$}
    \end{center}

    Горизонтальную составляющую магнитного поля Земли можно найти, используя период крутильных колебаний.

    \begin{equation*}
        T = 2 \pi \sqrt{\frac{J_n}{P_{mn} B_{||}}}
    \end{equation*}
    где $J_n$ -- момент инерции стрелки из $n$ шариков, $P_{mn} = P_m \cdot n$ -- магнитный момент стрелки.
    Момент инерции стрелки приближенно можно считать
    \begin{equation*}
        J_n \approx \frac{1}{3} n^3 m R^2
    \end{equation*} 
    тогда
    \begin{equation*}
        T = 2 \pi \sqrt{\frac{m R^2}{3 P_m B_{||}}} n
    \end{equation*}

    Снимем зависимость и построим график $T(n)$, тогда угловой коэффициент наклона будет равен $k = 2 \pi \sqrt{\frac{m R^2}{3 P_m B_{||}}}$.
    От сюда найдем горизонтальную составляющую магнитного поля Земли: $B_{||} = \frac{m R^2}{3 P_m k^2}$.

    \begin{table}[h!]
        \begin{center}
            \begin{tabular}{|c|c|c|c|c|c|}
                \hline
                $n$ & 11   & 10   & 9   & 8    & 7    \\ \hline
                $T$ & 2,86 & 2,67 & 2,4 & 2,23 & 1,95 \\ \hline
            \end{tabular}
        \end{center}
    \end{table}

    \begin{figure}[h!]
        \centering
        \includegraphics[scale = 1]{T_n_.jpg}
        \caption{}
    \end{figure}

    \begin{center}
        \fbox{$B_{||} = 19,6 $ мкТл}
    \end{center}

    Измерить вертикальную составляющую магнитного поля Земли можно с помощью той же установки, используя уравнение моментов.

    \begin{equation*}
        m g r_{\text{гр}} = n P_m B_{\perp}
    \end{equation*}

    \begin{center}
        \fbox{$B_{\perp} = 47 $ мкТл}
    \end{center}

    Найдем польный модуль магнитного поля Земли на текущей широте.

    \begin{equation*}
        B_0 = \sqrt{B_{||}^2 + B_{\perp}^2} = 44,2 ~ \text{мкТл}
    \end{equation*}

    Исследуем индукцию соленоида. Параметры шайбы: $d = 9$ мм, $h = 4$ мм.

    \begin{table}[h!]
        \begin{center}
            \begin{tabular}{|c|c|c|c|c|c|c|c|c|}
                \hline
                $n$      & 1   & 2   & 3   & 4   & 5   & 6   & 7   & 8   \\ \hline
                $B$, мТл & 232 & 314 & 349 & 355 & 362 & 363 & 364 & 369 \\ \hline
            \end{tabular}
        \end{center}
    \end{table}

    \begin{figure}[h!]
        \centering
        \includegraphics[scale = 0.5]{индукция соленоида.png}
        \caption{}
    \end{figure}

    Магнитное поле в произвольной точке $A$ на оси соленоида расчитывается по формуле

    \begin{equation*}
        B_A = \frac{\mu_0}{4 \pi} 2 \pi i (\cos \alpha - \cos \beta)
    \end{equation*}

    Для точки $O$ на торце соленоида $\cos \beta = 0$, так что для соленоида высотой $h$, радиусом $R$, и магнитным моментом
    $P_m$ поле на торце расчитывается по формуле

    \begin{equation*}
        B(h) = \frac{\mu_0}{2} P_m \frac{h}{\sqrt{R^2 + h^2}}
    \end{equation*}

    Проведем небольшое исследование функции $B(h)$.

    \begin{equation*}
        \lim_{h \rightarrow \infty} \frac{\mu_0}{2} P_m \frac{h}{\sqrt{R^2 + h^2}} = \frac{\mu_0}{2} P_m
    \end{equation*}

    Таким образом, график функции $B(h)$ должен иметь горизонтальную асимптоту $B_0 = \frac{\mu_0}{2} P_m$, что мы и можем
    наблюдать на практике.

    \newpage
    \begin{figure}[h!]
        \centering
        \includegraphics[scale = 1]{solenoid.png}
        \caption{}
    \end{figure}

\end{document}