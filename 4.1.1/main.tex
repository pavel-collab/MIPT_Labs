\documentclass[a4paper, 12pt]{article}

%%% Работа с русским языком
\usepackage{cmap}					% поиск в PDF
\usepackage{mathtext} 				% русские буквы в формулах
\usepackage[T2A]{fontenc}			% кодировка
\usepackage[utf8]{inputenc}			% кодировка исходного текста
\usepackage[russian]{babel}	% локализация и переносы

%%% Дополнительная работа с математикой
\usepackage{amsmath,amsfonts,amssymb,amsthm,mathtools} % AMS
\usepackage{icomma} % "Умная" запятая: $0,2$ --- число, $0, 2$ --- перечисление

%% Номера формул
%\mathtoolsset{showonlyrefs=true} % Показывать номера только у тех формул, на которые есть \eqref{} в тексте.

%% Шрифты
\usepackage{euscript}	 % Шрифт Евклид
\usepackage{mathrsfs} % Красивый матшрифт

%% Поля
\usepackage[left=2cm,right=2cm,top=2cm,bottom=2cm,bindingoffset=0cm]{geometry}

%% Русские списки
\usepackage{enumitem}
\makeatletter
\AddEnumerateCounter{\asbuk}{\russian@alph}{щ}
\makeatother

%%% Работа с картинками
\usepackage{graphicx}  % Для вставки рисунков
\graphicspath{{images/}{images2/}}  % папки с картинками
\setlength\fboxsep{3pt} % Отступ рамки \fbox{} от рисунка
\setlength\fboxrule{1pt} % Толщина линий рамки \fbox{}
\usepackage{wrapfig} % Обтекание рисунков и таблиц текстом

%%% Работа с таблицами
\usepackage{array,tabularx,tabulary,booktabs} % Дополнительная работа с таблицами
\usepackage{longtable}  % Длинные таблицы
\usepackage{multirow} % Слияние строк в таблице

%% Красная строка
\setlength{\parindent}{2em}

%% Интервалы
\linespread{1}
\usepackage{multirow}

%% TikZ
\usepackage{tikz}
\usetikzlibrary{graphs,graphs.standard}

%% Верхний колонтитул
\usepackage{fancyhdr}
\pagestyle{fancy}

%% Перенос знаков в формулах (по Львовскому)
\newcommand*{\hm}[1]{#1\nobreak\discretionary{}
	{\hbox{$\mathsurround=0pt #1$}}{}}

%% Мои дополнения
\usepackage{float} %Добавляет возможность работы с командой [H] которая улучшает расположение на странице
\usepackage{gensymb} %Красивые градусы
\usepackage{graphicx}               % Импорт изображений
\usepackage{caption} % Пакет для подписей к рисункам, в частности, для работы caption*

\begin{document}
	\begin{center}
    
    \normalsize{Федеральное государственное автономное образовательное учреждение высшего образования}
    
    \textbf{НАЦИОНАЛЬНЫЙ ИССЛЕДОВАТЕЛЬСКИЙ УНИВЕРСИТЕТ \\ <<МОСКОВСКИЙ ФИЗИКО-ТЕХНИЧЕСКИЙ ИНСТИТУТ>>}
    \vspace{13ex}
    
    \textbf{Лабораторная работа 3.1.3\\Измерение магнитного поля Земли.}
    \vspace{40ex}
    
    \normalsize{Филиппенко Павел Сергеевич \\ студент группы Б01-001\\ 2 курс ФРКТ\\}
\end{center}
    
\vfill 
    
\begin{center}
г. Долгопрудный\\ 
2021 г.
\end{center}


\thispagestyle{empty} % выключаем отображение номера для этой страницы
\newpage
	
	\section*{Теоретическая справка}
	\subsection*{Определение фокусных расстояний положительных и отрицательных линз и положений главных плоскостей сложной оптической системы}
	
	Оптическую систему называют \textbf{центрированной}, если центры всех поверхностей лежат на одной прямой, которую называют \textbf{главной оптической осью системы}. 
	
	Световые пучки называются \textbf{гомоцентрическими}, если, выйдя из одной точки и пройдя оптическую систему, пучки или их продолжения снова сходятся в одной точке.
	
	Идеальной оптической системой называют систему, в которой сохраняется \textbf{гомоцентричность} пучков и изображение геометрически подобно
	предмету
	
	Общий вид сложной оптической системы изображен на рис. \ref{fig:optical_system}.
	
	\begin{wrapfigure}{l}{0.45\textwidth}
		\includegraphics[width = 0.45\textwidth]{images/optical_system.png}
		\caption{сложная оптическая система}
		\label{fig:optical_system}
	\end{wrapfigure}

	Пусть $MM_1$ и $NN_1$ -- крайние поверхности, ограничивающие оптическую систему, а $O_1O_2$ -- главная оптическая ось. Точку $F_2$ называют \textbf{задним фокусом системы} (фокусом в пространстве изображений). Плоскость, перпендикулярная $O_1O_2$ и проходящая
	через точку $F_2$, называется \textbf{задней фокальной плоскостью}. Точку $F_1$ называют \textbf{передним фокусом системы} (фокусом в пространстве предметов). 
	
	Продолжим теперь $C_1D_1$ и $C_2D_2$ до пересечения с продолжениями $A_1B_1$ и $A_2B_2$ и отметим точки пересечения $R_1$ и $R_2$. Плоскости $P_1$ и $P_2$ называются \textbf{главными плоскостями}, а точки $H_1$ и $H_2$ -- \textbf{главными точками системы}. Расстояния от главных точек до фокусов называются \textbf{фокусными расстояниями}: $f_1 = H_1F+1$, $f_2 = H_2F_2$. В том случае, когда с обеих сторон системы находится одна и та же среда (например, воздух), $f_1 = f_2 = f$.
	
	\begin{wrapfigure}{r}{0.45\textwidth}
		\includegraphics[width = 0.45\textwidth]{images/image_in_thick_lense.png}
		\caption{Построение изображения в сложной оптической системе}
		\label{fig:image_in_thick_lense}
	\end{wrapfigure}

	Соотношения между величинами при построении изображения в толстой линзе удовлетворяют уравнению:
	
	\begin{equation} \label{eq:thin_lense_eq}
		\frac{1}{a_1} + \frac{1}{a_2} = \frac{1}{f}
	\end{equation}
	
	\subsubsection*{Способы определения фокусного расстояния тонкой положительной линзы}
	
	\begin{enumerate}
		\item Исходя из формулы тонкой линзы \eqref{eq:thin_lense_eq}, полагая $\delta = 0$.
		
		\item Пусть расстояние между предметом и экраном превышает $4f$. При этом всегда найдутся два таких положения линзы, при которых на экране получаются отчётливые изображения предмета (в одном случае уменьшенное, в другом -- увеличенное).
		
		\begin{figure}
			\centering
			\includegraphics[scale=0.5]{images/second_method.png}
			\caption{Нахождение фокусов тонкой собирающей линзы}
			\label{fig:second_method}
		\end{figure}
	
		Из соображений симметрии $a_1 = a_2'$ и $a_1' = a_2$. Тогда путем математических преобразований можно получить:
		
		\[ a_1 = \frac{L - l}{2}; ~~ a_2 = \frac{L + l}{2} \Rightarrow \]
		
		откуда с помощью \eqref{eq:thin_lense_eq} получаем:
		
		\begin{equation} \label{eq:second_method}
			f = \frac{L^2 - l^2}{4L}
		\end{equation}
		
		\item Фокусное расстояние тонкой положительной линзы можно определить с помощью зрительной трубы, настроенной на бесконечность, то есть на параллельный пучок лучей.
	\end{enumerate}

	\subsubsection*{Способы определения фокусного расстояния тонкой отрицательной линзы}
	
	\begin{enumerate}
		\item С помощью вспомогательной собирающей линзы получаем действительное изображение и используем формулу тонкой линзы.
		
		\begin{figure}
			\centering
			\includegraphics[width = 0.45\textwidth]{images/image_in_negative_lense.png}
			\caption{Изображение в рассеивающей линзе}
			\label{fig:image_in_negative_lense.png}
		\end{figure}
	
		\item С помощью зрительной трубы, установленной на бесконечность.
	\end{enumerate}

	\subsection*{Определение фокусного расстояния и положения главных плоскостей сложной оптической системы}
	
	Фокусное расстояние толстой положительной линзы определяют по методу Аббе рис. \ref{fig:Abbe_method}.
	
	\begin{figure}
		\centering
		\includegraphics[scale=0.5]{images/Abbe_method.png}
		\caption{метод Аббе}
		\label{fig:Abbe_method}
	\end{figure}

	Линейные увеличения для разных положений объекта:
	
	\[ \Gamma_1 = \frac{y_1}{y} = \frac{f}{x_1}; ~~ \Gamma_2 = \frac{y_2}{y} = \frac{f}{x_2} \]
	
	тогда нетрудно получить соотношение:
	
	\begin{equation} \label{eq:thik_lense_focus}
		f = \frac{\Delta x}{1 / \Gamma_1 - 1 / \Gamma_2}
	\end{equation}
	
	Для нахождения главных плоскостей системы недостаточно знать фокусное расстояние, нужно определить ещё положения главных фокусов. Это можно сделать при помощи зрительной трубы, настроенной на бесконечность.
	
	Теоретически фокусное расстояние $f_0$ сложной системы, состоящей из двух тонких положительных линз, можно рассчитать, если известны фокусные расстояния каждой линзы и расстояние между их центрами $l_{12}$:
	
	\begin{equation} \label{eq:theoretical_thik_lense_focus}
		\frac{1}{f_0} = \frac{1}{f_1} + \frac{1}{f_2} - \frac{|l_{12}|}{f_1 f_2}
	\end{equation}
	
	\subsection*{Недостатки (аберрации) реальных оптических систем}
	
	В идеальных оптических системах лучи, вышедшие из одной точки объекта, пересекаются в одной и той же точке изображения независимо от угла испускания и от длины волны света. В реальных системах -- из-за несовершенства линз -- такая зависимость наблюдается. Основными погрешностями линз являются сферическая и хроматическая аберрации.
	
	\subsection*{Сферическая аберрация}
	
	Сферическая аберрация возникает при преломлении широких (не параксиальных) пучков свет а на сферических поверхностях линз.
	
	\begin{wrapfigure}{l}{0.45\textwidth}
		\includegraphics[width = 0.45\textwidth]{images/spherical_aberration.png}
		\caption{Сферическая аберрация}
		\label{fig:spherical_aberration}
	\end{wrapfigure}

	Сферическую аберрацию характеризуют с помощью так называемой \textbf{продольной аберрации} $\delta s$ равной расстоянию между точками пересечения крайних и центральных лучей с главной оптической осью.
	
	Для лучей, проходящих на расстоянии $h$ от центра линзы, расстояние $s$ выражается соотношением:
	
	\begin{equation}
		s(h) = \frac{R}{n - 1} \left(1 - \frac{n^2 h^2}{2 R^2} \right)
	\end{equation}

	\begin{figure}
		\centering
		\includegraphics[scale=0.5]{images/aberation_equation.png}
		\caption{Сферическая аберрация плосковыпуклой линзы}
		\label{fig:aberration_equation}
	\end{figure}

	\textbf{Характеристической кривой} сферической аберрации называют зависимость
	
	\begin{equation} \label{eq:sphere_aberration}
		\delta s(h) = s(h) - s(0) = - \frac{n^2 h^2}{2(n - 1) R} = - \frac{1}{2} \left( \frac{n}{n - 1} \right)^2 \left( \frac{h}{f} \right)^2 f
	\end{equation}
	
	При $h = r$ ($r$ -- радиус линзы) формула \eqref{eq:sphere_aberration} определяет продольную сферическую аберрацию линзы.
	
	\subsection*{Хроматическая аберрация}
	
	Хроматическая аберрация (зависимость фокусного расстояния линзы от длин волны) возникает вследствие дисперсии показателя преломления стёкол.
	
	\begin{wrapfigure}{l}{0.45\textwidth}
		\includegraphics[width = 0.45\textwidth]{images/Chromatic_aberration.png}
		\caption{Хроматическая аберрация}
		\label{Chromatic_aberration}
	\end{wrapfigure}

	Хроматическую аберрацию принято характеризовать разностью фокусных расстояний для двух характерных спектральных линий водорода, расположенных в крайних частях видимой области спектра: $\lambda_F = 486.1$ нм (голубая линия $F$ водорода), $\lambda_C = 656.3$ (красная линия $C$ водорода):
	
	\begin{equation} \label{eq:chromatic_aberation}
		\delta f_{xp} = f_F - f_C
	\end{equation}

	Для характеристики дисперсионных свойств стёкол часто пользуются так называемым \textbf{коэффициентом дисперсии}, или \textbf{числом Аббе}:
	
	\begin{equation} \label{eq:Abbe_constant}
		\nu = \frac{n_D - 1}{n_F - n_C}
	\end{equation}
	
	где $n_F$ и $n_C$ -- показатели преломления для линий $F$ и $C$ водорода, а $n_D$ показатель преломления для жёлтой линии $D$ натрия $\lambda_D = 589.3$ нм (среднее значение длины волны жёлтого дублета натрия). Можно выразить продольную хроматическую аберрацию линзы через число Аббе:
	
	\begin{equation}
		\delta f_{xp} = - \frac{n_F - n_C}{n_D - 1} f_D = - \frac{1}{\nu} f_D
	\end{equation}
	
	\newpage
	
	\section*{Проведение эксперимента и обработка данных}
	
	\begin{enumerate}
		\item Настроим зрительную трубу на бесконечность и отцентрируем все компоненты оптической схемы.
		\item Отберем из представленных линз положительные путем наблюдения действительного изображения от лампы на потолке, на глаз определим их фокусные расстояния.
		
		Из представленных линз собирающими являются линзы № 1, 2 и 3, причем $F_1 \sim 10 \text{см}$, $F_2 \sim 10 \text{см}$, $F_3 = 0.8 \text{см}$.
		
		\item Измерим фокусное расстояние линзы № 1 методом Аббе.
		
		Используя метод Аббе фокусное расстояние можно посчитать следующим образом:
		
		\begin{equation}
			f = \frac{\Delta x}{\Delta (y/y')} = - \frac{\Delta x'}{\Delta (y'/y)}
		\end{equation}
		
		где $\Delta x$ -- смещение предмета, $\Delta x'$ -- смещение изображения, $\Delta (y'/y)$ -- приращение увеличения, $\Delta (y/y')$ -- приращение величины, обратной увеличению.
		
		\begin{equation}
			\Delta \left( \frac{y'}{y} \right) = \frac{y'_2}{y} - \frac{y'_1}{y}
		\end{equation} 
		
		Погрешности величин посчитаем по формулам косвенных погрешностей:
		
		\begin{equation}
			\sigma_i = \sqrt{\varepsilon_{y'_i}^2 + \varepsilon_{y}^2} \cdot \frac{y'_i}{y}; ~~ \sigma_{\Delta (y'/y)} = \sqrt{\sigma_1^2 + \sigma_2^2}
		\end{equation}
		
		тогда погрешность фокусного расстояния:
		
		\begin{equation}
			\sigma_f = \sqrt{\varepsilon_{\Delta x}^2 + \varepsilon_{\Delta (y'/y)}^2} \cdot f
		\end{equation}
		
		Результаты эксперимента:
		
		\begin{center}
			$y = 20.0 \pm 0.5$ мм -- размер предмета \\
			$a_1 = 27.70 \pm 0.05$ см -- расстояние от линзы до предмета в первом случае \\
			$a_2 = 45.50 \pm 0.05$ см -- расстояние от линзы до предмета во  втором случае \\
			$b_1 = 17.00 \pm 0.05$ см -- расстояние от линзы до экрана в первом случае \\
			$b_2 = 13.40 \pm 0.05$ см -- расстояние от линзы до экрана во втором случае \\
			$y_1 = 14.0 \pm 0.5$ мм -- размер изображения в первом случае \\
			$y_2 = 6.5 \pm 0.5$ мм -- размер изображения во втором случае \\
		\end{center}
	
		тогда получим, что 
		
		\begin{center}
			$\Delta x = 17.80 \pm 0.05$ см -- смещение предмета \\
			$\Delta x' = 3.60 \pm 0.05$ см -- смещение экрана \\
		\end{center}
	
		\[ \Delta \left( \frac{y'}{y} \right) = \frac{6.5 \text{мм}}{20 \text{мм}} - \frac{14 \text{мм}}{20 \text{мм}} = -0.375 \pm 0.04 \]
		
		\[ f = \frac{3.6}{0.375} = 9.6 \pm 1.03 ~ \text{см} \]
		
		Считая фокусное расстояние через $\Delta x$ получим:
		
		\[ f = 10.8 \pm 0.26 ~ \text{см} \]
		
		\item Найдем фокусное расстояние тонкой отрицательной линзы. Для того, чтобы получить на экране действительное изображение воспользуемся вспомогательной положительной линзой рис. \ref{fig:image_in_negative_lense.png}.
		
		Результаты измерений:
		
		\begin{center}
			$a_0 = 26.00 \pm 0.05$ см \\
			$l = 15.00 \pm 0.05$ см \\
			$a_2 = 59.10 \pm 0.05$ см \\
			$a_1 = a_0 - l = 11.00 \pm 0.05$ см \\
		\end{center}
	
		Определим фокусное расстояние, используя формулу тонкой линзы
		
		\begin{equation}
			\frac{1}{f} = \frac{1}{a_2} - \frac{1}{a_1} \Rightarrow f = \frac{a_1 a_2}{a_1 - a_2}
		\end{equation}
		
		Вычисление косвенных погрешностей: совершенно очевидно, что для некоторой ненулевой величины $x$ выполняется $\displaystyle \varepsilon_{1/x} = \varepsilon_x \Rightarrow \sigma_{1/x} = \frac{\varepsilon_x}{x}$. Тогда
		
		\begin{equation}
			\sigma_{1/f} = \sqrt{\left (\frac{\sigma_{a_1}}{a_1} \right)^2 + \left (\frac{\sigma_{a_2}}{a_2} \right)^2} \Rightarrow \sigma_f = \varepsilon_{1/f} \cdot f
		\end{equation}
		
		\[ f = \frac{11 \cdot 59.2}{11 - 59.1} \approx -13.52 \pm 0.85 ~ \text{см} \]
		
		\item Измерим фокусные расстояния линз с помощью зрительной трубы.
		
		Полученные величины:
		
		\begin{center}
			$f_{11} = 13.00 \pm 0.03$ см -- фокусное расстояние линзы № 1 с одной стороны \\
			$f_{12} = 11.30 \pm 0.03$ см -- фокусное расстояние линзы № 1 с другой стороны \\
			$f_{21} = 14.25 \pm 0.03$ см -- фокусное расстояние линзы № 2 с одной стороны \\ 
			$f_{22} = 14.50 \pm 0.03$ см -- фокусное расстояние линзы № 2 с другой стороны \\
			$|f_{31}| = 12.50 \pm 0.05$ см -- фокусное расстояние рассеивающей линзы с одной стороны \\
			$|f_{32}| = 13.25 \pm 0.05$ см -- фокусное расстояние рассеивающей линзы с другой стороны \\
		\end{center}
	
		По результатам измерений можно установить, можно ли считать данные линзы тонкими в рамках текущего эксперимента.
		
		Для линзы № 1: $f_{11} - f_{12} = 1.70 \pm 0.03 \sim 2 ~ \text{см} \Rightarrow$ данную линзу нельзя считать тонкой. Для линзы № 2 $f_{22} - f_{12} = 0.25 \pm 0.03 ~ \text{см} \Rightarrow$ данную линзу можно считать тонкой. Для линзы № 3 $|f_{32}| - |f_{31}| = 0.75 \pm 0.05 ~ \text{см} \Rightarrow$ данную линзу можно считать тонкой.
		
		\item Определение фокусного расстояния и главных плоскостей сложной оптической системы. Соберем установку, изображенную на рисунке \ref{fig:Abbe_method}. Запишем параметры установки:
		
		\begin{center}
			$l_{12} = 9.20 \pm 0.05$ см -- расстояние между линзами \\
			$a_1 = 6.45 \pm 0.05$ см -- расстояние от Л$_1$ до предмета в первом случае \\
			$b_1 = 62.30 \pm 0.05$ см -- расстояние от Л$_2$ до экрана в первом случае \\
			$a_2 = 8.15 \pm 0.05$ см -- расстояние от Л$_1$ до предмета во втором случае \\
			$b_2 = 29.60 \pm 0.05$ см -- расстояние от Л$_2$ до экрана во втором случае \\
			$y'_1 = 26.00 \pm 0.05$ см -- размер изображения в первом случае \\
			$y'_2 = 5.90 \pm 0.05$ см -- размер изображения во втором случае \\
		\end{center}
	
		Для расчета фокусного расстояния сложной системы воспользуемся формулой \eqref{eq:thik_lense_focus}. 
		
		\begin{center}
			$\displaystyle \Gamma_1 = \frac{y'_1}{y} = 13.00 \pm 0.33$ \\
			$\displaystyle \Gamma_2 = \frac{y'_2}{y} = 2.95 \pm 0.08$ \\
			$\Delta x = a_2 - a_1 = 1.70 \pm 0.05$ см \\
		\end{center}
	
		\[ f_{\text{эксп}} = \frac{1.7}{\frac{1}{2.95} - \frac{1}{13}} \approx 6.49 \pm 0.29 ~ \text{см} \]
		
		\[ f_{\text{теор}} = \frac{1}{\frac{1}{9.6} + \frac{1}{14.25} - \frac{9.2}{9.6 \cdot 14.25}} \approx 9.34 \pm 0.14 ~ \text{см} \]
		
		Теоретическое значение фокусного расстояния найдем по формуле \eqref{eq:theoretical_thik_lense_focus}.
		
		Найдем положение главных фокусов системы с помощью оптической трубы.
		
		\begin{center}
			$F_{01} = 41 \pm 0.5$ мм \\ 
			$F_{02} = 38 \pm 0.5$ мм \\
		\end{center}
	
		\item Изучение сферической аберрации. Нанесем на график \ref{fig:spheric_aberration_graph} зависимость $s(h^2)$, где $h$ -- диаметр диафрагмы. Экстраполируя график к $h^2 = 625$ получим
		
		\[ \delta s(r) = 3.51 \pm 0.2 ~ \text{мм} \]
		
		\begin{figure}
			\centering
			\includegraphics[scale=0.5]{images/spheric_aberration.pdf}
			\caption{График сферической аберрации}
			\label{fig:spheric_aberration_graph}
		\end{figure}
		
		% Откуда найдем показатель преломления $n = 1.5 \pm 0.1$.
		
		\item Изучение хроматической аберрации. Используя зрительную трубу и светофильтры измерим фокусные расстояния линзы на длинах волн $\lambda_F = 486.1$ нм (голубая линия $F$ водорода), $\lambda_C = 656.3$ (красная линия $C$ водорода). Используя формулу \eqref{eq:chromatic_aberation}, определим величину хроматической аберрации.
		
		\[ \delta f_{xp} = 0.3 \pm 0.1 ~ \text{мм} \] 
		
		\section*{Вывод}
		
		В проделанной работе мы изучили несколько способов определения фокусных расстояний тонки линз, способ нахождения фокусных расстояний и главных плоскостей сложной оптической системы, пронаблюдали эффекты сферической и хроматической аберрации. 
		
		Кроме того, мы убедились, что не все из предложенных линз можно считать тонкими в рамках данной работы.
		
		В результате проведения эксперимента мы убедились, что определение фокусных расстояний линз с помощью оптической трубы дает более точную величину.
		
		В опыте по измерению фокусных расстояний сложной оптической системы величина, полученная экспериментальным путем не совпала с теоретическим расчетом. Ошибки связаны предположительно с неточным измерением величин, грубой настройкой, а так же с тем фактом, что линзу № 1 нельзя считать тонкой в рамка данного эксперимента.  
		
	\end{enumerate}
	
\end{document}