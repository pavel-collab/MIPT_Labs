\input{preambule_article.tex}
% \newcommand{\R}{\mathbb{R}} % множество действительных чисел
% \newcommand{\N}{\mathbb{N}} % множество натуральных чисел
% \newcommand{\series}{\sum\limits_{k=1}^{\infty}} % ряд
% \newcommand{\useries}{\sum\limits_{k=1}^{\infty} u_k} % ряд u_k
% \newcommand{\useriesl}{\sum\limits_{k=1}^{\infty} u_k < \infty} % сходящийся ряд u_k
% \newcommand{\useriese}{\sum\limits_{k=1}^{\infty} u_k = \infty} % расходящийся ряд u_k
% \newcommand{\auseries}{\sum\limits_{k=1}^{\infty} |u_k|} % ряд модулей |u_k|
% \newcommand{\auseriesl}{\sum\limits_{k=1}^{\infty} |u_k| < \infty} % сходящийся ряд модулей |u_k| 
% \newcommand{\auseriese}{\sum\limits_{k=1}^{\infty} |u_k| = \infty} % расходящийся ряд модулей |u_k|

% \newcommand{\defeq}{\stackrel{def}{=}} % по определению
% \newcommand{\defarr}{\stackrel{def}{\Rightarrow}} % следует из определения

% \renewcommand {\geq}{\geqslant} % больше или равно
% \renewcommand {\leq}{\leqslant} % меньше или равно

\begin{document}
    \begin{center}
    
    \normalsize{Федеральное государственное автономное образовательное учреждение высшего образования}
    
    \textbf{НАЦИОНАЛЬНЫЙ ИССЛЕДОВАТЕЛЬСКИЙ УНИВЕРСИТЕТ \\ <<МОСКОВСКИЙ ФИЗИКО-ТЕХНИЧЕСКИЙ ИНСТИТУТ>>}
    \vspace{13ex}
    
    \textbf{Лабораторная работа 3.1.3\\Измерение магнитного поля Земли.}
    \vspace{40ex}
    
    \normalsize{Филиппенко Павел Сергеевич \\ студент группы Б01-001\\ 2 курс ФРКТ\\}
\end{center}
    
\vfill 
    
\begin{center}
г. Долгопрудный\\ 
2021 г.
\end{center}


\thispagestyle{empty} % выключаем отображение номера для этой страницы
\newpage

    \section*{Задание 1. Делитель напряжения.}

    \begin{figure}[h!]
        \centering
        \includegraphics[scale = 1]{делитель_напряжения.png}
        \caption{Делитель напряжения}
        %\label{}
    \end{figure}

    \noindent Напряжение питания $E = 10$ V, а выходное напряжение $E^* = 2$ V. 
    Возмем $R_1 = 6.8$ kOm, тогда вычислим $R_2$ по формуле:
    \begin{equation*}
        \frac{E - E^*}{R_1} = \frac{E^*}{R_2}
    \end{equation*} 
    Таким образом, получаем $R_2 = 1.6$ kOm.

    \noindent Измеряем полученное выходное напряжение при помощи АЦП в
    генераторе. Получаем $E^*_{\text{изм}} = 1.9$ V.

    \noindent Чтобы померить эквивалентное сопротивление источника, используем метод двух нагрузок. 
    Возьмём резистор $R_l = 6.8$ kOm. 
    При помощи того же АЦП меряем напряжение на нагрузке $U_l = 1.6$ V.
    Найдем эквивалентное сопративление по формуле:
    \begin{equation*}
        \frac{E^* - U_l}{R^*} = \frac{U_l}{R_l}
    \end{equation*} 
    Отсюда получаем $R^* \approx 1.29 $ kOm.

    \noindent Подаём на вход синусоидальное напряжение $e$. 
    Найдём коэффициент передачи:
    \begin{equation*}
        K = \frac{u}{e}
    \end{equation*} 
    В результате получаем, что эффективные $e = $ V, $u = $ V, то есть $K = $.
    Теоретический расчет дает $K_{\text{теор}} = $.

    \section*{Задание 2. Паралелльный сумматор.}

    \begin{figure}[h!]
        \centering
        \includegraphics[scale = 1]{сумматор.png}
        \caption{Паралелльный сумматор}
        %\label{}
    \end{figure}

    \noindent Из условия $\alpha = 0.4$, $\beta = 0.2$.

    \[ \frac{R_2}{R_1} = \frac{\alpha}{\beta} \]
    \[ \alpha + \beta = 0.6 = \frac{1}{1 + \frac{R_1 || R_2}{R}} \]

    \noindent Таким образом, получаем: $R_1 : R_2 : R = 1 : 2 : 1$.

    \noindent Возмем $R_1 = 5.1$ kOm. Собираем схему и смотрим на осциллографе постоянную и 
    переменную составляющую (либо поочередно закорачиваем источники). 
    Получаем, что $U_{=} = 1.03$ V, $U_{\approx} = 0.794$ V. Отсюда вычисляем $\alpha = 0.397$, $\beta =  0.206$.

    \noindent Измеряем по эквивалентное сопротивление по методу 2 нагрузок.
    Возьмём за нагрузку резистор $R_l = 5.1$ kOm. Тогда полученное напряжение
    $U^l_{=} = 0.731$ V, $U^l_{\approx} = 0.568$ V.
    Итоговое сопротивление получается равным $R^* = 2.06$ kOm. Расчетное сопративление
    $R^*_{\text{расч}} = R_1 || R_2 || R = 2.03$ kOm.

    \section*{Задание 3. H-параметры.}

    \noindent Проверим теоритическую зависимость. Если $U_2 = 0$, то
    \[ h_{11} = R_1 + R_2 || R_3 \]
    \[ h_{21} = \frac{R_3}{R_2 + R_3} \]
    Если $I_1 = 0$, то
    \[ h_{12} = \frac{U_1}{U_2} = - \frac{R_3}{R_2 + R_3} \]
    \[ h_{22} = \frac{I_2}{U_2} = \frac{1}{R_2 + R_3} \] 

    Проведем измерения в Micro-Cap.
    \begin{figure}[h!]
        \centering
        \includegraphics[scale = 1]{Hparams.png}
        \caption{H-параметры}
        %\label{}
    \end{figure}

    \[ h_{11} = \frac{U_1}{I_1} = \frac{2.2 ~ V}{1 ~ mA} = 2.2 ~ kOm \]
    \[ h_{12} = \frac{I_2}{I_1} = \frac{-600 ~ mkA}{1 ~ mA} = -0.6 \]
    \[ h_{21} = \frac{U_1}{U_2} = \frac{600 ~ mV}{1 ~ V} = 0.6 \]
    \[ h_{22} = \frac{I_1}{U_2} = \frac{200 ~ mkA}{1 mA} = 0.2 ~ kOm^{-1} \]

    \noindent А так как сопротивление резисторов $R_1 = 1$ kOm, $R2 = 2$ kOm,
    $R3 = 3$ kOm, то легко проверить, что теоритическая связь даёт тот же
    результат.

    \section*{Задание 4. Звезда и Треугольник.}

    \noindent Верность теоретической зависимость аналогично предыдущему 
    заданию можно найти из закона Ома в случаях, когда $I_1 = 0$ и $I_2 = 0$.
    Если сопротивления звезды равны $R_1 = 1$ kOm, $R_2 = 2$ kOm, 
    $R_3 = 3$ kOm, то пересчитав в параметры треугольника получим:
    $R_{13} = 5.5$ kOm, $R_{12} = 11/3$ kOm, $R_{23} = 11$ kOm.

    \newpage
    
    \noindent Проведём измерения в Micro-Cap.
    \begin{figure}[h!]
        \centering
        \includegraphics[scale = 1]{triangle.png}
        \caption{Звезда и треугольник}
        %\label{}
    \end{figure}

    \[ X_{11} = \frac{U_1}{I_1} = \frac{4 ~ V}{1 ~ mA} = 4 ~ kOm \]
    \[ X_{12} = \frac{U_1}{I_2} = \frac{3 ~ V}{1 ~ mA} = 3 ~ kOm \]
    \[ X_{21} = \frac{U_2}{I_1} = \frac{3 ~ V}{1 ~ mA} = 3 ~ kOm \]
    \[ X_{22} = \frac{U_2}{I_2} = \frac{5 ~ V}{1 ~ mA} = 5 ~ kOm \]

    \noindent Зная значения резистров, нетрудно проверить справедливость полученных данных.

    \begin{equation*}
        \left(
        \begin{array}{c}
            U_1\\
            U_2
        \end{array}
        \right) =
        \left(
        \begin{array}{cc}
            X_{11} & X_{12}\\
            X_{21} & X_{22}
        \end{array}
        \right)
        \left(
        \begin{array}{c}
            I_1\\
            I_2
        \end{array}
        \right) =
        \left(
        \begin{array}{cc}
            R_1 + R_3 & R_3      \\
            R_3       & R_2 + R_3
        \end{array}
        \right)
        \left(
        \begin{array}{c}
            I_1\\
            I_2
        \end{array}
        \right) 
    \end{equation*}

    \section*{Задание 5. Лестничные структуры.}

    \noindent Рассмотрим лестничные структуры.
    \begin{figure}[h!]
        \centering
        \includegraphics[scale = 1]{лестничные_структуры.png}
        \caption{Лестничные структуры.}
        %\label{}
    \end{figure}

    \noindent Исследуем напряжения в узлах и токи в ветвях для различных случаев.

    \begin{enumerate}
        \item $\alpha = 2,  ~ \gamma = \frac{1}{2}, ~ \omega = 2 ~ kOm$
        \item $\alpha = 6,  ~ \gamma = \frac{2}{3}, ~ \omega = 3 ~ kOm$
        \item $\alpha = 12, ~ \gamma = \frac{3}{4}, ~ \omega = 4 ~ kOm$
        \item $\alpha = 1,  ~ \gamma = \frac{\sqrt{5} - 1}{\sqrt{5} + 1} \approx 0.38, ~ \omega = \frac{1 + \sqrt{5}}{2} \approx 1.618 ~ kOm$
    \end{enumerate}

    \section*{Задание 5. ЦАП.}

    \noindent Снимем зависимость напряжения OUT от двоичного кода ($X_3$, $X_2$, $X_1$, $X_0$).

    \begin{enumerate}
        \item $(0, 0, 0, 0) = 0 ~ V$
        \item $(0, 0, 0, 1) = 1 ~ V$
        \item $(0, 0, 1, 0) = 2 ~ V$
        \item $(0, 0, 1, 1) = 3 ~ V$
        \item $(0, 1, 0, 0) = 4 ~ V$
        \item $(1, 0, 0, 0) = 8 ~ V$
        \item $(1, 1, 0, 1) = 13 ~ V$
        \item $(1, 1, 1, 1) = 15 ~ V$
    \end{enumerate}

\end{document}