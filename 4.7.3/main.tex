\documentclass[a4paper, 12pt]{article}

%%% Работа с русским языком
\usepackage{cmap}					% поиск в PDF
\usepackage{mathtext} 				% русские буквы в формулах
\usepackage[T2A]{fontenc}			% кодировка
\usepackage[utf8]{inputenc}			% кодировка исходного текста
\usepackage[russian]{babel}	% локализация и переносы

%%% Дополнительная работа с математикой
\usepackage{amsmath,amsfonts,amssymb,amsthm,mathtools} % AMS
\usepackage{icomma} % "Умная" запятая: $0,2$ --- число, $0, 2$ --- перечисление

%% Номера формул
%\mathtoolsset{showonlyrefs=true} % Показывать номера только у тех формул, на которые есть \eqref{} в тексте.

%% Шрифты
\usepackage{euscript}	 % Шрифт Евклид
\usepackage{mathrsfs} % Красивый матшрифт

%% Поля
\usepackage[left=2cm,right=2cm,top=2cm,bottom=2cm,bindingoffset=0cm]{geometry}

%% Русские списки
\usepackage{enumitem}
\makeatletter
\AddEnumerateCounter{\asbuk}{\russian@alph}{щ}
\makeatother

%%% Работа с картинками
\usepackage{graphicx}  % Для вставки рисунков
\graphicspath{{images/}{images2/}}  % папки с картинками
\setlength\fboxsep{3pt} % Отступ рамки \fbox{} от рисунка
\setlength\fboxrule{1pt} % Толщина линий рамки \fbox{}
\usepackage{wrapfig} % Обтекание рисунков и таблиц текстом

%%% Работа с таблицами
\usepackage{array,tabularx,tabulary,booktabs} % Дополнительная работа с таблицами
\usepackage{longtable}  % Длинные таблицы
\usepackage{multirow} % Слияние строк в таблице

%% Красная строка
\setlength{\parindent}{2em}

%% Интервалы
\linespread{1}
\usepackage{multirow}

%% TikZ
\usepackage{tikz}
\usetikzlibrary{graphs,graphs.standard}

%% Верхний колонтитул
\usepackage{fancyhdr}
\pagestyle{fancy}

%% Перенос знаков в формулах (по Львовскому)
\newcommand*{\hm}[1]{#1\nobreak\discretionary{}
	{\hbox{$\mathsurround=0pt #1$}}{}}

%% Мои дополнения
\usepackage{float} %Добавляет возможность работы с командой [H] которая улучшает расположение на странице
\usepackage{gensymb} %Красивые градусы
\usepackage{graphicx}               % Импорт изображений
\usepackage{caption} % Пакет для подписей к рисункам, в частности, для работы caption*

\begin{document}
	\begin{center}
    
    \normalsize{Федеральное государственное автономное образовательное учреждение высшего образования}
    
    \textbf{НАЦИОНАЛЬНЫЙ ИССЛЕДОВАТЕЛЬСКИЙ УНИВЕРСИТЕТ \\ <<МОСКОВСКИЙ ФИЗИКО-ТЕХНИЧЕСКИЙ ИНСТИТУТ>>}
    \vspace{13ex}
    
    \textbf{Лабораторная работа 3.1.3\\Измерение магнитного поля Земли.}
    \vspace{40ex}
    
    \normalsize{Филиппенко Павел Сергеевич \\ студент группы Б01-001\\ 2 курс ФРКТ\\}
\end{center}
    
\vfill 
    
\begin{center}
г. Долгопрудный\\ 
2021 г.
\end{center}


\thispagestyle{empty} % выключаем отображение номера для этой страницы
\newpage
	
	\textbf{Цель работы:} Ознакомление с методами получения и анализа поляризованного света.
	
	\textbf{В работе используются:}
	\begin{itemize}
		\item оптическая скамья с осветителем
		\item зелёный светофильтр
		\item два поляроида
		\item чёрное зеркало
		\item полированная эбонитовая пластинка
		\item стопа стеклянных пластинок
		\item пластинки в $1/4$ и $1/2$ длины волны
		\item пластинка в одну длину волны для зелёного света (пластинка чувствительного оттенка)
	\end{itemize}
	
	\section*{Теоретические положения}
	
	При помощи специальных приспособлений (поляризаторов), естественный свет может быть превращен в линейно поляризованный (или, как иногда говорят, в плоскополяризованный). В линейно поляризованной световой волне пара векторов \textbf{E} и \textbf{H} не изменяет с течением времени своей ориентации. Плоскость \textbf{E, S} называется в этом случае \textit{плоскостью колебаний}.
	
	Наиболее общим типом поляризации является \textit{эллиптическая поляризация}. В эллиптически поляризованной световой волне конец вектора
	\textbf{E} (в данной точке пространства) описывает некоторый эллипс. Линейно
	поляризованный свет можно рассматривать как частный случай эллиптически поляризованного света, когда эллипс поляризации вырождается в отрезок прямой линии; другим частным случаем является круговая поляризация (эллипс поляризации является окружностью).
	
	Для получения линейно поляризованного света применяются специальные оптические приспособления -- поляризаторы. Направление колебаний электрического вектора в волне, прошедшей через поляризатор, называется разрешенным направлением поляризатора. Всякий поляризатор может быть использован для исследования поляризованного света, т. е. в качестве анализатора. Интенсивность $I$ линейно поляризованного света после прохождения через анализатор за-
	висит от угла, образованного плоскостью колебаний с разрешенным направлением анализатора:
	\begin{equation} \label{eq:Malus}
	I = I_0 \cos 2\alpha.  
	\end{equation}
	
	Соотношение \eqref{eq:Malus} носит название \textit{закона Малюса}.
	
	Отраженный от диэлектрика свет всегда частично поляризован. Степень поляризации света, отраженного от диэлектрической пластинки в воздух, зависит от показателя преломления диэлектрика $n$ и от угла падения $i$. Как следует из формул Френеля, полная поляризация отраженного света достигается
	при падении под \textit{углом Брюстера}, который определяется соотношением
	
	\begin{equation}
	\tan i = n.   
	\end{equation}
	
	В этом случае плоскость колебаний электрического вектора в отраженном свете перпендикулярна плоскости падения. Для увеличения степени поляризации преломлённого света используют стопу стеклянных пластинок, расположенных под углом Брюстера к падающему свету.
	
	\textbf{Определение направления разрешенной плоскости колебаний поляроида} 
	При падении света на отражающую поверхность под углом Брюстера свет в отражённом луче полностью поляризован, а вектор \textbf{E} параллелен
	отражающей поверхности («правило иголки»). Луч света, прошедший поляроид и отразившийся от чёрного зеркала, имеет минимальную интенсивность при выполнении двух условий: во-первых, свет падает на отражающую поверхность под углом Брюстера, и во-вторых, в падающем пучке вектор
	\textbf{E} лежит в плоскости падения.
	
	Вращая поляроид вокруг направления луча и чёрное зеркало вокруг оси, перпендикулярной лучу, методом последовательных приближений
	можно добиться минимальной яркости луча, отражённого от зеркала, и таким образом определить разрешённое направление поляроида. Измеряя угол поворота зеркала (угол Брюстера), нетрудно определить коэффициент преломления материала, из которого изготовлено зеркало.
	
	\section*{Ход работы и обработка данных}
	
	\subsection*{Определение разрешённых направлений поляроидов}
	
	\begin{enumerate}
		\item Разместим на оптической скамье осветитель
		$S$, поляроид $P_1$ и чёрное зеркало. Поворачивая поляроид вокруг направления луча, добьёмся наименьшей яркости отражённого пятна. Оставим поляроид в этом положении и вращением зеркала вокруг вертикальной оси снова добьёмся минимальной интенсивности отражённого луча.
		
		\begin{figure}
			\centering
			\includegraphics[width=0.5\textheight]{./images/I.png}
			\caption{}
		\end{figure}
		
		Для первого поляроида разрешённое направление горизонтальное, на лимбе $\alpha_1 \approx 78^{\circ} ~ \pm ~ 2^{\circ}$.
		
		\item Вместо чёрного зеркала поставим второй поляроид. Скрестим их, определим разрешённое направление второго поляроида -- горизонтальная волна, на лимбе $\alpha_2 \approx 36^{\circ} ~ \pm ~ 2^{\circ}$.
	\end{enumerate}

	\subsection*{Определение угла Брюстера для эбонита}
	\begin{enumerate}
		\item Поставим на скамью вместо чёрного зеркала эбонитовую пластину с круговой шкалой.
		
		\item  Повернем эбонитовое зеркало вокруг вертикальной оси так, чтобы его
		плоскость была перпендикулярна лучу, начальная метка отсчета на шкале $\alpha_0 = 214^{\circ}$.
		
		\item Установим направление разрешённых колебаний поляроида $P_1$ горизонтально и найдите угол поворота эбонита $\varphi_{\text{Б}}$, при котором интенсивность отражённого луча минимальна: его абсолютное значение равно $\varphi_{\text{Б}} = 273^{\circ} - 214^{\circ} = 59^{\circ}$.
		
		\item Повторим измерения, добавив светофильтр. Угол Брюстера в этом эксперименте оказался равным $\varphi_{\text{Б}} = 56^{\circ}$.
		
		\item По углу Брюстера рассчитаем показатель преломления.
		\begin{equation}
			n = \tan \varphi_{\text{Б}} \Rightarrow n = 1.66
		\end{equation}
		Табличное значение показателя преломления эбонита $n = 1.6$.
		
	\end{enumerate}

	\subsection*{Исследование стопы}
	
	\begin{figure}
		\centering
		\includegraphics[width=0.5\textheight]{./images/III.png}
		\caption{}
	\end{figure}
	
	Поставим стопу стеклянных пластинок вместо эбонитового зеркала и подберем для неё такое положение, при котором свет падает на стопу под углом Брюстера. Осветим стопу неполяризованным светом и, рассматривая через поляроиды свет, отражённый от стопы, определим ориентацию вектора
	\textbf{E} в отражённом луче и в преломлённом луче.
		
	\begin{center}
		Для отраженного света $\gamma_1 = 85^{\circ}$ \\
		Для прошедшего света $\gamma_2 = 171^{\circ}$
	\end{center} 

	\subsection*{Определение главных плоскостей двоякопреломляющих пластин}
	
	\begin{figure}
		\centering
		\includegraphics[width=0.5\textheight]{./images/IV.png}
		\caption{}
	\end{figure}
	
	Поставим кристаллическую пластинку между скрещенными поляроидами. Вращая пластинку вокруг направления луча и наблюдая за интенсивностью света, про ходящего сквозь второй поляроид, определим, при каком условии главные направления пластинки совпадают с разрешёнными направлениями поляроидов. Повторим опыт для второй пластинки.
	
	\begin{center}
		Минимумы на первой пластинке наблюдаются при углах:\\
		$\alpha_1 = 12^{\circ} ~ \alpha_2 = 102^{\circ} ~ \alpha_3 = 190^{\circ} ~ \alpha_4 = 282^{\circ}$\\
		Минимумы на второй пластинке наблюдаются при углах:\\
		$\alpha_1 = 44^{\circ} ~ \alpha_2 = 134^{\circ} ~ \alpha_3 = 224^{\circ} ~ \alpha_4 = 314^{\circ}$\\
	\end{center}
	
	\subsection*{Выделение пластин $\lambda/2$ и $\lambda/4$}
	Добавим к схеме зелёный фильтр; установим разрешённое направление поляроида горизонтально, а главные направления исследуемой пластинки — под углом $45^{\circ}$ к горизонтали.
	
	Рассмотрим эффект от двух представленных пластинок:
	
	\begin{enumerate}
		\item Изучении первой пластинки: при вращении второго поляроида слегка (но не особо сильно) меняется цвет, а интенсивность света остается неизменной, то есть мы наблюдаем эллиптическую (или круговую поляризацию).
		\item Изучение второй пластинки: при вращении второго поляроида меняется интенсивность света, то есть мы наблюдаем линейную поляризацию.
	\end{enumerate} 

	Нам известно, что пластинка $\lambda/2$ не меняет характер поляризации, при её повороте, а пластинка $\lambda/4$ создаёт сдвиг фаз $\pi/2$ между колебаниями. Поскольку до прохождения пластинки свет был поляризован линейно (после прохождения через первый поляроид) изменение поляризации на эллиптическую (исчезновение минимумов при наблюдении) детектирует пластинку $\lambda/4$, сохранение минимумов (сохранение линейной поляризации) детектирует пластинку $\lambda/2$.
	
	\noindent \textbf{Вывод:} наблюдаемая пластинка №1 -- $\lambda/4$, пластинка №2 -- $\lambda/2$.
	
	\subsection*{Определение направлений большей и меньшей скоростей в пластинке $\lambda/4$}
	
	\begin{figure}
		\centering
		\includegraphics[width=0.5\textheight]{./images/VI.png}
		\caption{}
	\end{figure}
	
	\begin{enumerate}
		\item Поставим между скрещенными поляроидами пластинку чувствительного оттенка ($\lambda$ для зелёного света), имеющую вид стрелки. Световой вектор, ориентированный вдоль направления стрелки, проходит с большей скоростью, перпендикулярный -- с меньшей.
		
		Установим разрешённое направление первого поляроида горизонтально и убедимся с помощью второго поляроида, что эта пластинка
		не меняет поляризацию зелёного света в условиях предыдущего опыта.
		
		\item Уберем зелёный фильтр и поставим между скрещенными поляроидами
		пластинку $\lambda$ (стрелка под углом $45^{\circ}$ к разрешённым направлениям поляроидов). Глядя сквозь второй поляроид на стрелку, убедимся, что она имеет пурпурный цвет (зелёный свет задерживается вторым поляроидом, а красная и синяя компоненты проходят).
		
		\item . Добавим к схеме пластинку $\lambda/4$, главные направления
		которой совпадают с главными направлениями пластины $\lambda$ и ориентированы под углом $45^{\circ}$ к разрешённым направлениям скрещенных поляроидов
		
		\item Теперь уберём пластину чувствительного оттенка. После второго поляроида интенсивность минимальная -- значит, быстрая ось пластинки направлена горизонтально, направление вращения правое, направление колебаний в первом и третьем квадрантах (разность фаз $\pi/4$). При повороте рейтера со стрелкой на $180^{\circ}$ вокруг вертикальной оси
		цвет стрелки меняется от зелёно-голубого до оранжево-жёлтого.
		
	\end{enumerate}
	
	\subsection*{Определение направления вращения светового вектора в эллиптически поляризованной волне}
	
	\begin{figure}
		\centering
		\includegraphics[width=0.5\textheight]{./images/Эллипс.png}
		\caption{}
	\end{figure}
	
	\begin{enumerate}
		\item  Снова поставим зелёный фильтр, а за ним между скрещенными поляроидами -- пластинку произвольной толщины ($\lambda/4$).
		
		\item Получим эллиптически-поляризованный свет. Для этого установим разрешённое направление первого поляроида под углом $10-20^{\circ}$ к горизонтали так, чтобы вектор \textbf{E} падающего на пластинку света был расположен в первом квадранте.
		Установим разрешённое направление второго поляроида вертикально и, вращая пластинку, найдем минимальную
		интенсивность света, прошедшего второй поляроид. Вращая второй поляроид, убедитесь, что свет поляризован эллиптически,
		а не линейно.
		Таким образом, получим эллипс поляризации с вертикально ориентированной малой осью.
		\item  Для определения направления вращения светового вектора в эллипсе
		установим между поляроидами дополнительную пластинку $\lambda/4$ с известными направлениями <<быстрой>> и <<медленной>> осей, ориентированными по осям эллипса поляризации анализируемого света.
		В этом случае вектор \textbf{E} на выходе будет таким, как если бы свет прошёл две пластинки $\lambda/4$: свет на выходе из второй пластинки будет линейно поляризован. Если пластинки поодиночке дают эллипсы, вращающиеся в разные стороны, то поставленные друг за другом, они скомпенсируют
		разность фаз, и вектор \textbf{E} на выходе останется в первом
		и третьем квадрантах. Если же световой вектор перешёл в смежные квадранты, значит, эллипсы вращаются в одну сторону. 
		
		После второго поляроида интенсивность света максимальна. Значит, две пластины усиливают друг друга, световой вектор перешёл в смежные квадранты, эллипсы вращаются в одну сторону.
		
	\end{enumerate}
	
	\subsection*{Интерференция поляризованных лучей}
	
	Расположим между скрещенными поляроидами мозаичную слюдяную пластинку. Она собрана из 4-х узких полосок слюды, лежащих по сторонам квадрата (две полоски $\lambda/4$ и по одной -- $\lambda/2$ и $3\lambda/4$).
	 
	Вращаем пластинку: изменяется интенсивность света с периодичностью $\pi/4$
	
	Вращаем второй поляроид: изменяется цвет пластинок также с периодичностью  $\pi/4$.
	
	\section*{Контрольные вопросы}
	
	\begin{enumerate}
		\item Во всех процессах взаимодействия света с веществом основную роль играет вектор напряжённости электрического поля \textbf{E}, поэтому его называют \textbf{световым вектором}.
		
		
		
		\item Из законов преломления Снеллиуса $\sin \varphi = n \sin \psi$. Угол Брюстера $\tan \varphi = n$. Подставляя первое уравнение во второе получаем $\cos \varphi = \sin \psi$, откуда получаем $\displaystyle \varphi + \psi = \frac{\pi}{2}$.
	\end{enumerate}

\end{document}