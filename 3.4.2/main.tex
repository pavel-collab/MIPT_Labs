\input{include/preambule_article.tex}

\begin{document}
    % \begin{center}
    
    \normalsize{Федеральное государственное автономное образовательное учреждение высшего образования}
    
    \textbf{НАЦИОНАЛЬНЫЙ ИССЛЕДОВАТЕЛЬСКИЙ УНИВЕРСИТЕТ \\ <<МОСКОВСКИЙ ФИЗИКО-ТЕХНИЧЕСКИЙ ИНСТИТУТ>>}
    \vspace{13ex}
    
    \textbf{Лабораторная работа 3.1.3\\Измерение магнитного поля Земли.}
    \vspace{40ex}
    
    \normalsize{Филиппенко Павел Сергеевич \\ студент группы Б01-001\\ 2 курс ФРКТ\\}
\end{center}
    
\vfill 
    
\begin{center}
г. Долгопрудный\\ 
2021 г.
\end{center}


\thispagestyle{empty} % выключаем отображение номера для этой страницы
\newpage
    % \newpage

    Запишем закон Кюри-Вейесса

    \begin{equation}
        \chi \propto \frac{1}{T - \theta_p}
    \end{equation}

    \noindent где $\theta_p$ -- параметр размерности температуры. Закон Кюри-Вейесса хорошо выполняется вдали от $\theta_K$ -- 
    температуры Кюри, однако нарушается при $T \rightarrow \theta_K$. Поэтому параметр $\theta_p$ отличается от температуры Кюри
    (как правило $\theta_K < theta_p$). На практике наблюдается картина, изображенная на \hyperref[theory_xi]{рисунке}.

    \begin{figure}
        \centering
        \includegraphics[scale = 0.75]{theory.png}
        \caption{Зависимость обратной магнитной восприимчивости от температуры}
        \label{theory_xi}
    \end{figure}

    Для практического исследования будем применять формулу

    \begin{equation}
        \frac{1}{\chi} \propto (T - \theta_p) \propto \frac{1}{\tau^2 - \tau_0^2}
    \end{equation}

    \noindent $\tau$ -- период колебания автогениратора, $\tau_0$ -- период колебаний автогениратора в отсутствии исследуемого
    образца.

    Запишем параметры установки

    \begin{center}
        $\tau_0 = 9,045$ мкс -- период колебаний автогениратора в отсутствии исследуемого образца \\
        $L_0 = 1602$ мкГн -- индуктивность в отсутствии исследуемого образца \\
        $k = 24$ град/мВ -- чувствительность термопары \\
    \end{center}

    Отсюда посчитаем точность определения температуры

    \[ 24 \cdot 3 \cdot 10^{-3} ~ \text{мВ} = 0,072 ~ C^{o} \]

    Выражение для индуктивности тороидальной катушки в СИ

    \begin{equation}
        L = \frac{\mu_0 \mu N}{2 \pi R} S
    \end{equation}

    Период колебаний по формуле Джоуля-Томсона

    \begin{equation}
        \tau = 2 \pi \sqrt{L C}
    \end{equation}

    \begin{equation}
        \frac{\tau^2 - \tau_0^2}{\tau_0^2} = \frac{\tau^2}{\tau_0^2} - 1 = \frac{\mu}{\mu_0} - 1 = \mu - 1 = \chi        
    \end{equation}

    Таким образом, получаем

    \begin{equation}
        \chi = \frac{\tau^2}{\tau_0^2} - 1
    \end{equation}

    \begin{table}[]
    \begin{center}
        \begin{tabular}{|c|c|c|c|c|c|c|}
            \hline
            $t, ~ C^o$ & $\tau$, мкс & $\chi$    & $1/\chi$  & $T, ~ K$ & $\mu$    & $L$, мкГн  \\ \hline
            10,77      & 10,86101    & 0,44186   & 2,26315   & 283,92   & 1,44186  & 2309,86073  \\ \hline
            12,12      & 10,8358     & 0,43517   & 2,29792   & 285,27   & 1,43517  & 2299,15012  \\ \hline
            14,14      & 10,76926    & 0,41760   & 2,39461   & 287,29   & 1,41760  & 2270,99978  \\ \hline
            16,13      & 10,66155    & 0,38938   & 2,56813   & 289,28   & 1,38938  & 2225,79962  \\ \hline
            18,12      & 10,4786     & 0,34211   & 2,92300   & 291,27   & 1,34211  & 2150,06650  \\ \hline
            20,11      & 10,17474    & 0,26540   & 3,76782   & 293,26   & 1,26540  & 2027,17858  \\ \hline
            22,1       & 9,79891     & 0,17364   & 5,75872   & 295,25   & 1,17364  & 1880,18640  \\ \hline
            24,1       & 9,52033     & 0,10786   & 9,27084   & 297,25   & 1,10786  & 1774,79981  \\ \hline
            26,1       & 9,39631     & 0,07918   & 12,6280   & 299,25   & 1,07918  & 1728,86086  \\ \hline
            30,09      & 9,27918     & 0,05245   & 19,0652   & 303,24   & 1,05245  & 1686,02716  \\ \hline
        \end{tabular}
    \end{center}
    \caption{Экспериментальные данные}
    \label{experiment_table}
\end{table}

    Используя данные, полученные при проведении эксперимента, построим графики зависимости \hyperref[graph_1x(T)]{$1/\chi(T)$},
    \hyperref[graph_x_mu_(T)]{$\chi(T)$}, \hyperref[graph_x_mu_(T)]{$\mu(T)$} и \hyperref[graph_L(T)]{$L(T)$}.

    \begin{figure}
        \centering
        \includegraphics[width = \linewidth]{graph_1x(T).png}
        \caption{Зависимость обратной магнитной восприимчивости от температуры}
        \label{graph_1x(T)}
    \end{figure}

    \begin{figure}
        \centering
        \includegraphics[width=\linewidth]{imgonline-com-ua-2to1-spGCn2gxrV.jpg}
        \caption{Графики зависимости магнитной восприимчивости и магнитной проницаемости от температуры}
        \label{graph_x_mu_(T)}
    \end{figure}

    \begin{figure}[h!]
        \centering
        \includegraphics[width=\linewidth]{graph_L(T).png}
        \label{graph_L(T)}
        \caption{Зависимость индуктивности от температуры}
    \end{figure}

    Используя эксперементальные данные и графики получаем значение параметра $\theta_p$:

    \begin{center}
        \fbox{$\theta_p = 291 ~ K = 18 ~ C^o$}
    \end{center}

\end{document}