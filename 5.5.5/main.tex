\documentclass[a4paper, 12pt]{article}

%%% Работа с русским языком
\usepackage{cmap}					% поиск в PDF
\usepackage{mathtext} 				% русские буквы в формулах
\usepackage[T2A]{fontenc}			% кодировка
\usepackage[utf8]{inputenc}			% кодировка исходного текста
\usepackage[russian]{babel}	% локализация и переносы

%%% Дополнительная работа с математикой
\usepackage{amsmath,amsfonts,amssymb,amsthm,mathtools} % AMS
\usepackage{icomma} % "Умная" запятая: $0,2$ --- число, $0, 2$ --- перечисление

%% Номера формул
%\mathtoolsset{showonlyrefs=true} % Показывать номера только у тех формул, на которые есть \eqref{} в тексте.

%% Шрифты
\usepackage{euscript}	 % Шрифт Евклид
\usepackage{mathrsfs} % Красивый матшрифт

%% Поля
\usepackage[left=2cm,right=2cm,top=2cm,bottom=2cm,bindingoffset=0cm]{geometry}

%% Русские списки
\usepackage{enumitem}
\makeatletter
\AddEnumerateCounter{\asbuk}{\russian@alph}{щ}
\makeatother

%%% Работа с картинками
\usepackage{graphicx}  % Для вставки рисунков
\graphicspath{{images/}{images2/}}  % папки с картинками
\setlength\fboxsep{3pt} % Отступ рамки \fbox{} от рисунка
\setlength\fboxrule{1pt} % Толщина линий рамки \fbox{}
\usepackage{wrapfig} % Обтекание рисунков и таблиц текстом

%%% Работа с таблицами
\usepackage{array,tabularx,tabulary,booktabs} % Дополнительная работа с таблицами
\usepackage{longtable}  % Длинные таблицы
\usepackage{multirow} % Слияние строк в таблице

%% Красная строка
\setlength{\parindent}{2em}

%% Интервалы
\linespread{1}
\usepackage{multirow}

%% TikZ
\usepackage{tikz}
\usetikzlibrary{graphs,graphs.standard}

%% Верхний колонтитул
\usepackage{fancyhdr}
\pagestyle{fancy}

%% Перенос знаков в формулах (по Львовскому)
\newcommand*{\hm}[1]{#1\nobreak\discretionary{}
	{\hbox{$\mathsurround=0pt #1$}}{}}

%% Мои дополнения
\usepackage{float} %Добавляет возможность работы с командой [H] которая улучшает расположение на странице
\usepackage{gensymb} %Красивые градусы
\usepackage{graphicx}               % Импорт изображений
\usepackage{caption} % Пакет для подписей к рисункам, в частности, для работы caption*

\begin{document}
    \begin{center}
    
    \normalsize{Федеральное государственное автономное образовательное учреждение высшего образования}
    
    \textbf{НАЦИОНАЛЬНЫЙ ИССЛЕДОВАТЕЛЬСКИЙ УНИВЕРСИТЕТ \\ <<МОСКОВСКИЙ ФИЗИКО-ТЕХНИЧЕСКИЙ ИНСТИТУТ>>}
    \vspace{13ex}
    
    \textbf{Лабораторная работа 3.1.3\\Измерение магнитного поля Земли.}
    \vspace{40ex}
    
    \normalsize{Филиппенко Павел Сергеевич \\ студент группы Б01-001\\ 2 курс ФРКТ\\}
\end{center}
    
\vfill 
    
\begin{center}
г. Долгопрудный\\ 
2021 г.
\end{center}


\thispagestyle{empty} % выключаем отображение номера для этой страницы
\newpage

    \section*{Цель работы}

    Снять и исследовать спектры излучения различных источников, характеризовать различные пики в спектрах радиоактивных веществ.

    \section*{Теоретическая чать}

    Основная задача спектрометрических измерений заключается в определении энергии, интенсивности дискретных гамма-линий от различных гамма-источников и их идентификации.
	
	Основными процессами взаимодействия гамма-излучения с веществом являются фотоэффект, эффект Комптона и образование электрон-позитронных пар. Каждый из этих процессов вносит свой вклад в образование наблюдаемого спектра. Образующиеся при этих процессах электроны испытывают большое количество неупругих соударений с молекулами и атомами среды. Неупругие соударения могут сопровождаться как ионизацией, так и возбуждением молекул или атомов среды. В промежуточных же стадиях (при переходах возбужденных молекул или атомов в основное состояние, при рекомбинации электрических зарядов и т.п.) в веществе возникают кванты света различных длин волн, присущих данному веществу.
	
	При \textbf{фотоэффекте} кинетическая энергия электрона $ T_e = E_{\gamma} - I_i $, где $ I_i $ --- энергия ионизации $ i $-той оболочки атома. Фотоэффект особенно существенен для тяжелых веществ, где он идет с заметной вероятностью даже при высоких энергиях гамма-квантов. В легких веществах фотоэффект становится заметен лишь при относительно небольших энергиях гамма-квантов. Наряду с фотоэффектом, при котором вся энергия гамма-кванта передается атомному электрону, взаимодействие гамма-излучения со средой может приводить к его рассеянию, т.е. отклонению от первоначального направления распространения на некоторый угол.
	
	При \textbf{эффекте Компотна} происходит упругое рассеяние фотона на свободном электроне, сопровождающееся изменением длины волны фотона (реально этот процесс происходит на слабо связанных с атомом внешних электронах). Максимальная энергия образующихся комптоновских электронов соответствует рассеянию гамма-квантов на $ 2\pi $ и равна
	
	\begin{equation}\label{E_compton}
	E_{с \_ max} = \dfrac{\hbar \omega}{1 + \dfrac{m_ec^2}{2\hbar\omega}}
	\end{equation}
	
	При достаточно высокой энергии гамма-кванта наряду с фотоэффектом и эффектом Комптона может происходить третий вид взаимодействия гамма-квантов с веществом – \textbf{образование электрон-позитронных пар}. При этом если процесс образования пары идет в кулоновском поле ядра или протона, то энергия образующегося ядра отдачи оказывается весьма малой, так что пороговая энергия гамма-кванта, необходимая для образования пары, практически совпадает с удвоенной энергией покоя электрона $ Е_0 = 2m_ec^2 =1,022  $МэВ.
	
	Появившийся в результате процесса образования пар электрон теряет свою энергию на ионизацию среды. Таким образом, вся энергия электрона остается в детекторе. Позитрон будет двигаться до тех пор, пока практически не остановится, а затем аннигилирует с электроном среды, в результате чего появятся два гамма-кванта. Т.е., кинетическая энергия позитрона также останется в детекторе. Далее возможны три варианта развития событий:
	
	а) оба родившихся гамма-кванта не вылетают из детектора, и тогда вся энергия первичного гамма-кванта останется в детекторе, а в спектре появится пик с $ E = E_\gamma $;
	
	б) один из родившихся гамма-квантов покидает детектор, и в спектре появляется пик, соответствующий энергии $  Е = Е_\gamma - E0 $, где $ Е_0 = m_ec^2 = $ 511 кэВ;
	
	в) оба родившихся гамма-кванта покидают детектор, и в спектре появля- ется пик, соответствующий энергии $  Е = Е_\gamma - 2E0 $, где $ 2Е_0 = 2m_ec^2 = $ 1022 кэВ;
	
	Таким образом, любой спектр, получаемый с помощью гамма-спектрометра, описывается несколькими компонентами, каждая из которых связана с определенным физическим процессом. Как описано выше, основными физическими процессами взаимодействия гамма-квантов с веществом являются фотоэффект, эффект Комптона и образование электрон-позитронных пар, и каждый из них вносит свой вклад в образование спектра. Помимо этих процессов, добавляются экспонента, связанная с наличием фона, пик характеристического излучения, возникающий при взаимодействии гамма-квантов с окружающим веществом, а также пик обратного рассеяния, образующийся при энергии квантов $ Е_\gamma \gg mc^22/2 $ в результате рассеяния гамма-квантов на большие углы на материалах конструктивных элементов детектора и защиты. Положение пика обратного рассеяния определяется по формуле ($ E $ --- энергия фотопика):
	
	\begin{equation}\label{Eobr}
		E_{обр} = \dfrac{E}{1 + \dfrac{2E}{mc^2}}
	\end{equation}

%	\section{Экспериментальная установка}
%	
	Энергетическим разрешением спектрометра называется величина
	
	\begin{equation}\label{Ri = dE/E}
	R_i = \dfrac{\Delta E_i}{E_i}
	\end{equation}
	
	т.е. отношение ширины пика полного поглощения (измеренной на полувысоте) к регистрируемой энергии пика поглощения. Это значение $ E_i \propto \overline{n_i} $ --- числу частиц на выходе ФЭУ. При этом  $ \Delta E_i \propto \overline{\Delta n_i} = \sqrt{\overline{n_i}} $ --- ширина пика пропорциональна среднеквадратичной флуктуации, которая равна корню из числа частиц. Таким образом, наша формула \eqref{Ri = dE/E} примет вид
	
	\begin{equation}\label{Ri = c/E}
	R_i = \dfrac{\mathrm{const}}{\sqrt{E_i}}
	\end{equation}

    \section*{Обработка эксперементальных данных}

    % значению канала соответсвует значение энергии гамма-кванта
    % нам известны значения энергий фотопиков для 3х изотопов
    % эксперементально мы можем узнать номера каналов фотопиков для этих изотопов
    % поскольку зависимость линейная, мы можем построить линейный график и однозначно определять энергию по значению канала

    % линейные коэффициенты kx + b:
    % k = 0.7469766723171799 +- 0.005710818813457874
    % b = -53.23185435309176 +- 7.109247697642382

    Определим энергии края комптоновского поглощения для Co, Cs и Na, и сравним с теоретическими значениями.

    \begin{table}[h!]
    \centering
    \begin{tabular}{|c|c|c|}
    \hline
       & exp  & thr  \\ \hline
    Co & 895  & 963  \\ \hline
    Cs & 425  & 477  \\ \hline
    Na & 1072 & 1062 \\ \hline
    \end{tabular}
    \caption{Край комптоновского рассеяния}
    \label{tab:kompton}
\end{table}

    определим энергии пиков обратного рассеяния и сравним их с теоретическими.

    \begin{table}[h!]
    \centering
    \begin{tabular}{|c|c|c|}
    \hline
       & exp & thr \\ \hline
    Co & 220 & 214 \\ \hline
    Cs & 197 & 184 \\ \hline
    \end{tabular}
    \caption{Энергии пиков обратного рассеяния}
    \label{tab:back_f}
\end{table}

\end{document}